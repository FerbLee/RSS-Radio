\chapter[Apéndice]{
  \label{chp:dicionario}
  Dicionario de datos
}
Neste apéndice figura, por orde alfabética, unha descrición dos modelos de datos utilizados. Inclúense so as entidades definidas de forma propia, non aquelas dadas polo framework (por exemplo, User). Do mesmo xeito, non se inclúen os atributos automaticamente definidos pola declaración relacións en Django. 

\textbf{Broadcast:} Clase que representa a relación N-N de emisión entre programas e emisoras.

\begin{longtable}{|p{3cm}|p{3cm}|p{8cm}|}
	\hline
	\rowcolor{gray!50}
	Atributo & Tipo & Descrición\\
	\hline
	@\textbf{id} & Serial & Clave primaria\\
	\hline
	\textbf{program} & ForeignKey & Clave foránea da entidade Program\\
	\hline
	\textbf{station} & ForeignKey & Clave foránea da entidade Station\\	
	\hline
	\textbf{schedule\_details} & Char(100) & Texto do comentario\\
	\hline
\end{longtable}


\textbf{Comment:} Entidade que garda os comentarios que os usuarios deixan nos episodios.

\begin{longtable}{|p{3cm}|p{3cm}|p{8cm}|}
	\hline
	\rowcolor{gray!50}
	Atributo & Tipo & Descrición\\
	\hline
	@\textbf{id} & Serial & Clave primaria\\
	\hline
	\textbf{episode} & ForeignKey & Clave foránea da entidade Episode\\
	\hline
	\textbf{user} & ForeignKey & Clave foránea da entidade User\\	
	\hline
	\textbf{text} & Text & Texto do comentario\\
	\hline
	\textbf{publication\_date} & DateTime & Data de publicación (GMT)\\
	\hline
	\textbf{removed} & Boolean & Marcado coma borrado\\
	\hline
\end{longtable}



\textbf{Episode:} Entidade que garda cada unha das entregas (episodios) dos programas.

\begin{longtable}{|p{3cm}|p{3cm}|p{8cm}|} 
	\hline
	\rowcolor{gray!50}
	Atributo & Tipo & Descrición\\
	\hline
	@\textbf{id} & Serial & Clave primaria\\
	\hline
	\textbf{program} & ForeignKey & Clave foránea da entidade Program\\
	\hline
	\textbf{title} & Char(200) & Título do episodio\\	
	\hline
	\textbf{summary} & Text & Resumo do episodio\\
	\hline
	\textbf{publication\_date} & DateTime & Data de publicación (GMT)\\
	\hline
	\textbf{insertion\_date} & DateTime & Data de inserción na base de datos (GMT)\\
	\hline
	\textbf{file} & URL & Enlace ao ficheiro de audio\\
	\hline
	\textbf{file\_type} & Char(40) & Enlace ao ficheiro de audio\\
	\hline
	\textbf{downloads} & BigInt & Contador de descargas e escoitas do episodio\\
	\hline
	\textbf{original\_id} & Char(200) & Id do episodio no sistema orixinal de almacenamento\\
	\hline
	\textbf{original\_site} & URL & Páxina do episodio no sistema orixinal de almacenamento\\
	\hline
	\textbf{removed} & Boolean & Marcado coma borrado\\
	\hline
	\textbf{image} & ForeignKey & Clave foránea da entidade Image\\
	\hline
	\textbf{votes} & ManyToMany & Referencia ás instancias de Vote relacionadas\\
	\hline
	\textbf{comments} & ManyToMany & Referencia ás instancias de Comment relacionadas\\
	\hline
	
\end{longtable}


\textbf{Program:} Entidade que garda os programas engadidos polos usuarios.

\begin{longtable}{|p{3cm}|p{3cm}|p{8cm}|}
	\hline
	\rowcolor{gray!50}
	Atributo & Tipo & Descrición\\
	\hline
	@\textbf{id} & Serial & Clave primaria\\
	\hline
	\textbf{image} & ForeignKey & Clave foránea da entidade Image\\
	\hline
	\textbf{name} & Char(200) & Nome do programa\\	
	\hline
	\textbf{description} & Text & Texto descritivo sobre o programa\\
	\hline
	\textbf{creation\_date} & DateTime & Data de inserción na base de datos (GMT)\\
	\hline
	\textbf{author\_email} & Email & Correo electrónico do autor do programa\\
	\hline
	\textbf{author} & Char(200) & Autor orixinal extraído do ficheiro RSS\\
	\hline
	\textbf{language} & Char(10) & Código de linguaxe do programa\\
	\hline
	\textbf{rss\_link} & URL & Enlace ao ficheiro RSS do programa\\
	\hline
	\textbf{rss\_link\_type} & Char[ivoox $|$ radioco $|$ podomatic] & Tipo de RSSLinkParser utilizado na súa creación\\
	\hline
	\textbf{rating} & PositiveSmall Integer[0:100] & Cualificación calculada para o programa\\
	\hline
	\textbf{original\_site} & URL & Enlace ao podcast orixinal\\
	\hline
	\textbf{popularity } & Float & Popularidade calculada para o programa\\
	\hline
	\textbf{website} & URL & Páxina web do programa\\
	\hline
	\textbf{sharing\_options} & Char[share\_free $|$ no\_share] & Condicións de compartición\\
	\hline
	\textbf{comment \_options} & Char[enable $|$ disable] & Opción de activar ou desactivar comentarios\\
	\hline
	\textbf{subscribers} &  ManyToMany & Referencia ás instancias de User relacionadas. Representa os subscritores do programa\\
	\hline
	\textbf{admins} &  ManyToMany & Referencia ás instancias de ProgramAdmin relacionadas\\
	\hline
\end{longtable}


\textbf{ProgramAdmin:} Clase que representa a relación N-N de administración entre programas e usuarios.

\begin{longtable}{|p{3cm}|p{3cm}|p{8cm}|}
	\hline
	\rowcolor{gray!50}
	Atributo & Tipo & Descrición\\
	\hline
	@\textbf{id} & Serial & Clave primaria\\
	\hline
	\textbf{program} & ForeignKey & Clave foránea da entidade Program\\
	\hline
	\textbf{user} & ForeignKey & Clave foránea da entidade User\\	
	\hline
	\textbf{type} & Char[owner $|$ admin] & Permisos do usuario sobre o programa\\
	\hline
	\textbf{date} & DateTime & Data na que se concedeu o permiso (GMT)\\
	\hline
\end{longtable}


\textbf{Station:} Entidade que garda os colectivos de emisión (radios por ondas, radios por internet, canles de podcast...)

\begin{longtable}{|p{3cm}|p{3cm}|p{8cm}|}
	\hline
	\rowcolor{gray!50}
	Atributo & Tipo & Descrición\\
	\hline
	@\textbf{id} & Serial & Clave primaria\\
	\hline
	\textbf{logo} & ForeignKey & Clave foránea da entidade Image para o logotipo da emisora\\
	\hline
	\textbf{profile\_img} & ForeignKey & Clave foránea da entidade Image para a imaxe de cabeceira do perfil\\
	\hline
	\textbf{name} & Char(200) & Nome da emisora\\	
	\hline
	\textbf{broadcasting \_method} & Char[RadioFM $|$ RadioAM $|$ RadioDigital $|$ TVChannel $|$ RadioInternet $|$ PodcastingChannel $|$ Others] & Método de emisión dos programas \\
	\hline
	\textbf{broadcasting \_area} & Char(200) & Área de emisión (No caso de RadioFM, RadioAM e RadioDigital)\\
	\hline
	\textbf{broadcasting \_frequency} & Char(50) & Frecuencia de emisión (No caso de RadioFM, RadioAM e RadioDigital)\\
	\hline
	\textbf{streaming\_link} & URL & Enlace á emisión en directo por streaming\\
	\hline
	\textbf{website} & URL & Enlace á páxina web do colectivo\\
	\hline
	\textbf{location} & Char(200) & Localización da emisora\\
	\hline
	\textbf{programs} &  ManyToMany & Referencia ás instancias de Broadcast relacionadas\\
	\hline
	\textbf{admins} &  ManyToMany & Referencia ás instancias de ProgramAdmin relacionadas\\
	\hline
	\textbf{followers} &  ManyToMany & Referencia ás instancias de User relacionadas. Representa os seguidores da emisora.\\
	\hline
\end{longtable}

\textbf{StationAdmin:} Clase que representa a relación N-N de administración entre emisoras e usuarios.

\begin{longtable}{|p{3cm}|p{3cm}|p{8cm}|}
	\hline
	\rowcolor{gray!50}
	Atributo & Tipo & Descrición\\
	\hline
	@\textbf{id} & Serial & Clave primaria\\
	\hline
	\textbf{station} & ForeignKey & Clave foránea da entidade Station\\
	\hline
	\textbf{user} & ForeignKey & Clave foránea da entidade User\\	
	\hline
	\textbf{type} & Char[owner $|$ admin] & Permisos do usuario sobre a emisora\\
	\hline
	\textbf{date} & DateTime & Data na que se concedeu o permiso (GMT)\\
	\hline
\end{longtable}	


\textbf{Tag:}  Entidade que garda os etiquetas de categoría dadas polos ficheiros RSS.

\begin{longtable}{|p{3cm}|p{3cm}|p{8cm}|}
	\hline
	\rowcolor{gray!50}
	Atributo & Tipo & Descrición\\
	\hline
	@\textbf{id} & Serial & Clave primaria\\
	\hline
	\textbf{name} & Char(50) & Nome do tag en minúsculas. Ten que ser único\\
	\hline
	\textbf{times\_used} & Positive IntegerField & Cantidade de veces presente en programas e episodios\\	
	\hline
	\textbf{programs} & ManyToMany & Referencia ás instancias de Program relacionadas\\
	\hline
	\textbf{episodes} & ManyToMany & Referencia ás instancias de Episode relacionadas\\
	\hline
\end{longtable}

\pagebreak
\textbf{UserProfile:}  Entidade que extende a clase User para asignarlle novos atributos.

\begin{longtable}{|p{3cm}|p{3cm}|p{8cm}|}
	\hline
	\rowcolor{gray!50}
	Atributo & Tipo & Descrición\\
	\hline
	@\textbf{id} & Serial & Clave primaria\\
	\hline
	\textbf{user} & OneToOne & Referencia á instancia de User que extende\\
	\hline
	\textbf{description} & Text & Texto de presentación do usuario\\	
	\hline
	\textbf{avatar} & ForeignKey & Clave foránea da entidade Image\\
	\hline
	\textbf{location} & Char(100) & Localización do usuario\\
	\hline
\end{longtable}


\textbf{Vote:}  Entidade que representa os votos cos que os usuarios cualifican os episodios.

\begin{longtable}{|p{3cm}|p{3cm}|p{8cm}|}
	\hline
	\rowcolor{gray!50}
	Atributo & Tipo & Descrición\\
	\hline
	@\textbf{id} & Serial & Clave primaria\\
	\hline
	\textbf{user} & ForeignKey & Clave Foránea da entidade User\\
	\hline
	\textbf{episode} & ForeignKey & Clave foránea da entidade Episode\\
	\hline
	\textbf{location} & Char(100) & Localización do usuario\\
	\hline
	\textbf{date} & DateTime & Data na que se concedeu o permiso (GMT)\\
	\hline
\end{longtable}

