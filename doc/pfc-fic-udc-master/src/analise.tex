\chapter[Análise]{
  \label{chp:analise}
  Análise
}
\minitoc
\newpage

Neste capítulo descríbese o funcionamento ideal da aplicación. Exporanse tamén os requirimentos funcionais e non funcionais identificados ao abordar os obxectivos do proxecto. 

\section{Descrición do funcionamento}

O sistema consiste nunha páxina web con apartados adicados, por un lado, ao que chamamos actividades de consumo, e por outro ás actividades de produción.

\subsection{Actividades de consumo de contidos}

Son aquelas \textbf{levadas a cabo polos ouvintes} das emisoras e os programas presentes no sistema. Un usuario, no seu papel de \say{consumidor}, pode, mediante a interface web, acceder ao catálogo de programas, xa sexa mediante as ferramentas de busca proporcionadas ou a través das recomendacións que a propia páxina amosa en portada.

Dentro da páxina correspondente a unha emisora, o usuario atopará un reprodutor HTML5 desde o que \textbf{escoitar a súa emisión en directo} mediante streaming. O usuario terá opción de \textbf{facerse \say{seguidor}} da emisora para acceder a ela de xeito máis rápido desde a páxina principal.

Tamén obterá acceso á lista de programas emitidos por dita emisora co seu horario de emisión. Na páxina de cada programa verá a lista completa de episodios dispoñibles e terá a opción de \textbf{subscribirse} para poder acceder de xeito máis rápido aos novos episodios publicados.

Na páxina propia de episodio atopará outro reprodutor HTML5 que lle posibilitará ou ben a escoita por streaming ou ben a \textbf{descarga directa do ficheiro de audio} correspondente. Poderá votar o programa a favor ou en contra (isto repercutirá na popularidade do programa, concepto explicado máis adiante) e deixar un comentario no caso de que esta opción esté habilitada polos administradores do programa.


\subsection{Actividades de produción de contidos}

Son aquelas levadas a cabo polos \textbf{propietarios e administradores} dos programas e emisoras. Un usuario, no seu papel de produtor, poderá engadir ao sistema unha \textbf{nova emisora} cubrindo o formulario habilitado a tal efecto na interface web no que deberá introducir manualmente os datos.

Poderá tamén \textbf{engadir un programa}. Para isto, o usuario só necesitará proporcionarlle ao sistema un enlace ao ficheiro de RSS do podcast que queira engadir e o sistema encargarase de extraer a información tanto dos programas coma dos episodios correspondentes. Este programa e os seus episodios manteranse actualizados xa que o propio backend do sistema comprobará periodicamente se houbo algún cambio no RSS e actualizará a base de datos de xeito acorde.  

Un usuario que posúa un programa ou emisora pode invitar a outro usuario a \textbf{colaborar na xestión} desta ou deste.


\section{Requirimentos funcionais}

Son aqueles que describen as \textbf{accións que o sistema debe efectuar}, isto é: a colección de capacidades e características que o software pon a disposición do usuario. Presentaranse estes en forma de \textbf{historias de usuario} (ver capítulo \ref{chp:metodoloxia} sobre a metodoloxía). 

Se ben é certo que houbo \textbf{encontros con membros de colectivos do terceiro sector} coma a URCM (Unión de radios libres e comunitarias de Madrid) e Cuac FM (A radio comunitaria da Coruña), parte delas están extraídas da \textbf{experiencia propia} coma colaborador nestes medios e coma moderador de comunidades en liña.


\begin{itemize}
	\item Ter un punto de encontro onde ver os programas que están a emitir as emisoras federadas na ReMC (Red de Medios Comunitarios)
	\item Que unha emisora poda acceder aos ficheiros de audio das demais.
	\item Que os \say{autores} dun programa poidan saber en que emisoras está a ser emitido e a que hora.
	\item Que os programas se poidan dar de alta e de baixa.
	\item Poder visualizar información das emisoras e dos programas: Nome, información de emisión, datos de contacto...
	\item Poder subscribirse aos programas.
	\item Poder compartir contido entre usuarios.
	\item Que os ficheiros de audio se asocien de xeito automático a cada emisora.
	\item Que non sexa necesario engadir manualmente os novos ficheiros.
	\item Que os usuarios podan colaborar na xestión das emisoras.
	\item Agrupar os programas por categorías.
\end{itemize}

Eses \say{ficheiros de audio} nos que se pensaba nunha primeira aproximación foron o que finalmente derivou no concepto de \say{episodios de programa} que se explicará máis a fondo no capítulo \ref{chp:disenho} sobre o deseño.

Gran parte destas historias veñen dos usuarios que producen o contido. A medida que avanzaba o deseño e os primeiros pasos da implementación, fóronse perfilando novas historias de usuario pensando máis nos ouvintes:

\begin{itemize}
	\item Que os ouvintes poidan manifestar a súa opinión sobre os programas mediante votos e comentarios.
	\item Que os ouvintes poidan manter unha lista de programas e emisoras favoritas.
	\item Ter ferramentas de procura de contidos.
	\item Recibir recomendacións.  
\end{itemize}  

E tamén outros requirimentos propios da xestión dos contidos.

\begin{itemize}
	\item Crear diferentes roles de administración.
	\item Que o administrador dun programa poida restrinxir a emisión do seu contido.
	\item Que o administrador dun programa poida moderar os comentarios no seu contido.
\end{itemize}  

\section{Requirimentos non funcionais}

Son aqueles que \textbf{especifican as propiedades do sistema} no referente á manexabilidade, seguridade e robustez\cite{softreq}.

\subsection{Autenticación}

O sistema ten que permitir o rexistro de usuarios cun login único e un contrasinal. Dito contrasinal ha de gardarse cifrado na base de datos. A información de sesión da autenticación debe protexerse de manipulacións para evitar impostores.

\subsection{Protección de datos}

Os usuarios deben saber en todo momento que datos persoais deles son recompilados polo sistema e para que fin serán utilizados. Ademáis, débese garantir o seguinte:

\begin{itemize}
	\item \textbf{Dereito ao esquecemento:} Os datos dun usuario han de ser borrados do sistema cando este así o requira.
	\item \textbf{Disponibilidade:} O usuario ha de poder acceder aos seus datos sen posibilidade de ocultación ou demora por parte do sistema. 
\end{itemize}


\subsection{Internacionalización}

O sistema debe ter soporte para a tradución a distintos idiomas. As datas deben gardarse nun formato común e zona horaria UTC(Universal Time Coordinated) para adecuarse na interface ás preferencias do usuario.

\subsection{Rendemento}

A carga de obxectos desde a base de datos debe realizarse de xeito \say{lazy} (preguiceiro) de modo que non se sobrecargue a memoria de forma innecesaria. Isto é especialmente eficiente en comprobacións de existencia de obxectos ou na navegación entre clases relacionadas. 

O tráfico entre o cliente e o servidor debe minimizarse mediante, por exemplo, un sistema de paxinación para aquelas vistas que amosen unha grande cantidade de datos.

\subsection{Concorrencia}

O servidor ten que ser capaz de servir á páxina a un número de usuarios simultáneos aceptable. Isto depende en grande medida do hardware no que o servidor sexa executado. En calquera caso, sendo as emisoras comunitarias colectivos pequenos e baseándonos en datos de visitas á web de Cuac FM, non se espera que o cumprimento deste requisito mereza especial atención.

\subsection{Seguridade}

Os novos datos que se engadan ao sistema han de ser validados para evitar comportamentos maliciosos coma ataques de inxección de código. Cómpre tamén facer validación das URL's para evitar accesos non permitidos. 

\subsection{Adaptabilidade a dispositivos móbiles}

Xa que nos capítulos anteriores mencionamos a importancia dos dispositivos móbiles no crecemento da radio por Internet, a páxina debe ser amigable con eles adaptando a interface cando a pantalla na que se execute sexa pequena.


\subsection{Usabilidade}

O sistema debe poder ser utilizado por usuarios cun perfil non técnico. As ferramentas de entrada e saída de datos han de ser o suficientemente comprensibles e coñecidas para o público xeral.
