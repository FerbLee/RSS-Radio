\thispagestyle{plain}
\section*{Resumo}

Este proxecto consiste no desenvolvemento dunha aplicación web que permite a escoita de gravacións de radio ou podcast.

O sistema conta con: emisoras, cuxa emisión en directo se pode escoitar mediante streaming, e programas, cuxos episodios quedan dispoñibles ou ben para escoitar por streaming baixo demanda, ou ben para a súa descarga directa. Estes contidos son engadidos e administrados polos propios usuarios, que poderán colaborar nas labores de xestión dos mesmos.

A web funciona tamén a modo de catálogo de programas para as emisoras. Unha emisora do sistema pode incorporar os programas existentes á súa emisión, sendo posible a difusión dun programa por parte de varios colectivos. 

A motivación para este traballo foi a de proporcionar un punto de encontro en liña para as radios libres e comunitarias co obxectivo aumentar a visibilidade dos seus programas en Internet e favorecer a compartición de contidos e a colaboración entre elas.

O resultado do proxecto é un produto de software libre. Para a súa creación utilizouse Django, un framework de desenvolvemento web de Python e empregáronse técnicas propias das metodoloxías áxiles. Esta memoria explica o proceso completo desde a súa concepción, pasando polas fases de análise, deseño, planificación e dando detalles da súa implementación e validación.




   
