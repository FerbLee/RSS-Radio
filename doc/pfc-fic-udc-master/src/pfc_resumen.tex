\thispagestyle{plain}
\section*{Resumo}

Nos últimos anos, Internet converteuse nunha peza clave para os medios de radiodifusión ao dar acceso centralizado e global a uns contidos que, mediante o xeito tradicional, só poderían ser escoitados onde as ondas cheguen e nunca máis aló de onde permita a licenza legal da que dispoña o medio. Isto é especialmente interesante para as pequenas emisoras locais, a miúdo comunitarias, culturais e con orzamento limitado.

A estas últimas está orientado este proxecto. Consiste nun punto de encontro online para promocionar contidos radiofónicos e favorecer a súa reemisión por parte de distintos medios de comunicación (Emisoras, canles de podcasting...) así coma o seu consumo directo por parte dos visitantes da web. Para o seu desenvolvemento, utilizouse Django 1.11, un framework web de Python, ferramentas HTML5, Javascript e CSS-grid. Tamén se utilizaron ferramentas de sindicación RSS para o acceso aos contidos de terceiros.

Nesta memoria tratarase o proceso completo de desenvolvemento do proxecto desde as fases de análise e deseño até os detalles de implementación. Mencionaranse tamén as liñas de traballo que se pretenden seguir no futuro.

   
