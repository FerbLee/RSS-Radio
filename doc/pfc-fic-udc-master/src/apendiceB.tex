\chapter[Apéndice]{
  \label{chp:dicionario}
  Licenza
}

Este apéndice ten o propósito de determinar a licenza baixo a que se distribúen este proxecto e maila súa documentación. Un dos obxectivos a priori era que o software resultante fose compatible coa definición de software libre que da a Free Software Foundation, segundo a cal, un programa ten que cumprir o que comunmente se chama \say{As catro liberdades esenciais}\cite{licenza}:

\begin{itemize}
	\item \textbf{Liberdade 0:} Liberdade de executar o programa con calquera propósito desexable.
	\item \textbf{Liberdade 1:} Liberdade para estudar o funcionamento do programa e de modificalo sen limitacións. Para garantir esta liberdade, o código fonte debe estar dispoñible ao acceso público.
	\item \textbf{Liberdade 2:} Liberdade de redistribución de copias do programa orixinal. 
	\item \textbf{Liberdade 3:} Liberdade de distribución de copias de versións modificadas do software sempre e cando ese código modificado estea á súa vez dispoñible ao público.
\end{itemize} 

\section{Licenzas das dependencias do proxecto}

A continuación móstrase unha táboa das dependencias do software deste proxecto. Centrarémonos en dúas características:

\begin{itemize}
	\item Liberdade: Se a licenza é compatible coa definición de software libre vista con anterioridade.
	\item Copyleft: Dise daquela licencia que esixa preservar as súas liberdades na distribución do produto ou derivados. Un software con licenza libre non copyleft podería ser utilizado noutro software de natureza privativa.
\end{itemize} 
\break

\begin{longtable}{|p{3cm}|p{1.5cm}|p{5.5cm}|p{1cm}|p{1.5cm}|}
	\hline
	\rowcolor{gray!50}
	Dependencia & Versión & Licenza & Libre & Copyleft \\
	\hline
	\textbf{Anaconda} & 4.4.0 (64bits) & BSD (Berkeley Software Distribution) & Si & Non \\
	\hline
	\textbf{Django} & 1.11 & BSD & Si & Non \\
	\hline
	\textbf{psycopg2} & 2.7.1 & LGPL (GNU Lesser General Public License) & Si & Si \\	
	\hline
	\textbf{django-bootstrap4} & 0.0.6 & Apache Software License 2.0 & Si & Si\\
	\hline
	\textbf{celery} & 4.1.0 & BSD & Si & Non \\
	\hline
	\textbf{django-celery-results} & 1.0.1 & BSD & Si & Non \\
	\hline
	\textbf{django-celery-beat} & 1.1.1 & BSD & Si & Non \\
	\hline
	\textbf{django-static-jquery} & 2.1.4 & MIT (Massachusetts Institute of Technology) & Si & Non \\
	\hline
	\textbf{gettext} & 0.19.8.1-3 & GPLv3 (GNU General Public License) & Si & Si \\
	\hline
\end{longtable}


\section{Conclusión}

O software creado neste proxecto publícase baixo a licenza \textbf{GPLv3}. Esta foi orixinalmente creada para o proxecto GNU pola propia Free Software Foundation, sendo a versión 3, publicada en 2007, a máis recente. Trátase dunha licencia copyleft, o cal significa que a distribución das copias deste software e calquera traballo derivado han de ser tamén un proxecto de software libre. Esta licenza permite non so que, no futuro, outros desenvolvedores podan colaborar na mellora e mantemento deste software, senón tamén que outros proxectos o reutilicen.

A documentación, é dicir, a presente memoria, queda publicada baixo licenza \textbf{GFDL} (GNU Free Documentation License). É unha licenza libre creada tamén pola FSF para o proxecto GNU. Este feito implica que o presente documento pode ser copiado e modificado por quen o desexe. Ao ser GFDL unha licenza copyleft, a redistribución de copias e obras derivadas desta documentación está suxeita ao mantemento dos termos da licencia por parte das mesmas.

