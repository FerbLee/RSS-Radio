% PFC: PhotoPlace
% (c) Jose Riguera Lopez 2013

% Opciones:
%       a4paper -> indica el tamaño del papel, en este caso A4.
%       11pt    -> tamaño de la fuente 11 puntos.
%       twoside -> doble cara.
%       oneside -> una cara del folio.
%
\documentclass[a4paper,11pt,twoside]{book}
%% Estilo PFC UDC
\usepackage{pfc}
%% Macros
\newcommand{\curso}{Enxeñaría en Informática}
\newcommand{\proyecto}{Proxecto Fin de Carreira}
\newcommand{\universidad}{Universidade de A Coruña}
\newcommand{\centro}{Facultade de Informática}
\newcommand{\unicentro}{\centro da \universidad}
\newcommand{\proyectotipo}{Proxecto clásico de enxeñaría}
\newcommand{\titulo}{Servicio en línea para la publicación de grabaciones de radio y podcasts}
\newcommand{\titulogalego}{Servizo en liña para a publicación de gravacións de radio e podcasts}
\newcommand{\tituloenglish}{Online service for publishing radio broadcasting recordings and podcasts}
\newcommand{\corto}{Servizo de publicación de gravacións de radio}
\newcommand{\autor}{Fernando Liñares Varela}
\newcommand{\director}{José María Casanova Crespo}
\newcommand{\tutor}{}
\newcommand{\departamento}{Departamento de Computación}
\newcommand{\software}{PhotoPlace}
\newcommand{\palabras}{Radio, Podcast, Web, Django, Postgres, Python, Javascript, jQuery, CSS Grid. }

\title{\titulo}
\author{\autor}
\date{\today}

\hypersetup{
%   bookmarks=true,         % barra de marcadores
    unicode=false,          % caracteres non-Latin en marcadores de Acrobat
    pdftoolbar=true,        % mostrar barra de herramientas de Acrobat
    pdfmenubar=true,        % mostrar menú de Acrobat
    pdffitwindow=false,     % ajustar ventana al ancho de página
    pdfstartview={FitH},    % ajustar documento al ancho de página
    pdftitle={\corto},      % título
    pdfauthor={\autor},     % autor
    pdfsubject={\titulo},   % tema del documento
    pdfcreator={LaTex},     % generador del documento
    pdfproducer={\autor},   % productor del documento
    pdfkeywords={\palabras},% lista de palabras clave
    pdfnewwindow=true,      % enlaces en una nueva ventana
    colorlinks=true,        % false: enlaces en caja; true: enlaces coloreados
    linkcolor=black,        % color de enlaces internos
    citecolor=green,        % color de enlaces a bibliografía
    filecolor=magenta,      % color de enlaces a ficheros
    urlcolor=blue           % color de enlaces externos
}

\loadglsentries{pfc.glossary}
\makeglossaries

\begin{document}
  \graphicspath{{./images/}}
  % pageblank genera una pagina en blanco, para imprimir en formato libro
  %
% Portada.
%
\begin{titlepage}
	\begin{center}
		% Logotipo de la universidad.
		\includegraphics[width=10cm]{./images/03-simbolo-logo-cor-udc-2.png}
		\vspace{2cm}

		% Nombre de la facultad, de la universidad y del departamento.
		{\Large{\textbf{\centro}}}
		\\
		{\it \large{\textbf{\departamento}}}
		\vspace{1cm}

		{\large {\sc \proyecto}\\{\curso}}
		\vspace{1cm}

		% Título
		\textbf{\Large \titulogalego}
		\vspace{6cm}
	\end{center}

	\begin{flushright}
		\begin{tabular}{ll}
			\large{\textbf{Alumno:}}	&
			\large{\autor} \\

			\large{\textbf{Director:}}	&
			\large{\director} \\

			%\large{\textbf{Tutor:}}		&
			%\large{\tutor} \\

			% Fecha.
			\large{\textbf{Data:}}		&
			\large{20 de xuño de 2018} \\
		\end{tabular}
	\end{flushright}
\end{titlepage}

  \pageblank
  %
% Certificado
%
\thispagestyle{plain}
\begin{center}
	\begin{minipage}[t][6cm][l]{.8\textwidth}
		\begin{center}
			% Director del proyecto
			D. {\sc \director}

			% Profesores
			Profesor, \centro

			% Departamento al que pertenece el director y en el que se realiza el proyecto.
			\departamento
			
			\universidad
		\end{center}
	\end{minipage}
\end{center}

CERTIFICA:

Que a memoria titulada {\it ``\titulogalego''} foi realizada por {\sc \autor}
conforme á descrición inicialmente proposta baixo a miña direción e
constitúe o seu {\proyecto} de {\curso}.
Pola presente, autorizo a súa presentación para que o Proxecto sexa defendido
nesta convocatoria.

\vspace{3cm}

En A Coruña, a \today

% Espacio para que pueda firmar el certificado que debe acompañar al proyecto.
\vspace{3cm}
\begin{center}
	\begin{minipage}[t][4cm][l]{.5\textwidth}
	D. {\sc \director}
	\\
	Director do proxecto
	\end{minipage}
\end{center}

  \pageblank
  %
% Nota
%
\thispagestyle{plain}
\section*{}


\begin{tabular}{p{2cm}p{11cm}}
	\large{Título} & \\
	& \textbf{\large{\titulogalego}} \\
	\\
	& \textbf{\large{\titulo}} \\
	\\
	& \textbf{\large{\tituloenglish}} \\
	\\
	\large{\textbf{Clase:}} & \large{\proyectotipo} \\
	\\
	\large{\textbf{Autor:}} & \large{\autor} \\
	%\large{\textbf{Tutor:}} & \large{\tutor} \\
	\large{\textbf{Director:}} & \large{\director} \\
	\large{\textbf{Data:}} & \large{\today} \\
	\\
	\large{\textbf{Tribunal}} & \\
	& \vspace{3cm} \\
	\\
	\large{\textbf{Data de defensa:}} & \\
	\\
	\large{\textbf{Calificación:}} & \\
	\\
\end{tabular}

  \pageblank
  \thispagestyle{plain}
\section*{Resumo}

Nos últimos anos, Internet converteuse nunha peza clave para os medios de radiodifusión xa que permite un alcance global e o acceso baixo demanda aos contidos emitidos. Isto é especialmente interesante para as pequenas emisoras locais, a miúdo comunitarias, culturais e con orzamento limitado.

A estas últimas está orientado este proxecto. Consiste nun punto de encontro online para promocionar contidos radiofónicos e favorecer a súa reemisión por parte de distintos medios de comunicación (Emisoras, canles de podcasting...) así coma o seu consumo directo por parte dos visitantes da web. Para o seu desenvolvemento, utilizouse Django 1.11, un framework web de Python, ferramentas HTML5, Javascript e CSS-grid. Tamén se utilizaron ferramentas de sindicación RSS para o acceso aos contidos de terceiros.

Nesta memoria tratarase o proceso completo de desenvolvemento do proxecto desde as fases de análise e deseño até os detalles de implementación. Mencionaranse tamén as liñas de traballo que se pretenden seguir no futuro.

   

  \pageblank
  \include{pfc_palabras_clave}
  \pageblank
  %
% Dedicatoria
%
\thispagestyle{plain}

\begin{flushright}
	{\it Aos meus compañeiros de prácticas.}
\end{flushright}

  \pageblank
  %
% Agradecimientos
%
\chapter*{Agradecimientos}
\vspace*{2cm}

A los profesores {\director} por sus consejos durante el desarrollo del proyecto.  POR HACER\\



  \pageblank
  
  % FRONTMATTER: TOC, LOF, LOT
  \frontmatter
  \dominitoc[n]     % removes the title "contents"
  \nomtcrule        % removes rules = horizontal lines
  \tableofcontents
  %\addcontentsline{toc}{chapter}{Contents}
  \listoffigures
  %\addcontentsline{toc}{chapter}{List of Figures}
  %\listoftables
  %\addcontentsline{toc}{chapter}{List of Tables}

  % MAINMATTER: content
  \mainmatter
  \chapter[Introdución]{
  \label{chp:introduccion}
  Introdución
}
\minitoc
\newpage

Nos últimos anos, Internet converteuse nunha peza clave para os medios de radiodifusión xa que permite un alcance global e o acceso baixo demanda aos contidos emitidos. Isto é especialmente interesante para as pequenas emisoras locais, a miúdo comunitarias, culturais e con orzamento limitado.

A estas últimas está orientado este proxecto. Consiste nun punto de encontro en liña para promover contidos radiofónicos e favorecer a súa redifusión por parte de distintos medios de comunicación (Emisoras, canles de podcasting...) así coma o seu consumo directo por parte dos visitantes da web. Para o seu desenvolvemento, utilizouse Django 1.11, un framework web de Python, ferramentas HTML5, Javascript e CSS-grid. Tamén se utilizaron ferramentas de sindicación RSS para o acceso aos contidos de terceiros.

Nesta memoria tratarase o proceso completo de desenvolvemento do proxecto desde as fases de análise e deseño até os detalles de implementación. Mencionaranse tamén as liñas de traballo que se pretenden seguir no futuro.

\section{Marco do proxecto}


A radiodifusión tradicional, entendendo esta coma a retransmisión de contidos de audio a través de ondas analóxicas, presenta a día de hoxe unha serie de limitacións. A máis importante, se cadra, é o feito de que a demanda de frecuencias é superior ao que o espectro radioeléctrico pode ofrecer. Cómpre, por iso, a existencia de unha autoridade que outorgue licenzas de emisión sendo ditas institucións, na nosa sociedade, a Secretaría de Estado de Telecomunicaciones y para la Sociedad de la Información e máis a Secretaría Xeral de Medios\cite{BOE}. Isto implica a imposibilidade de emitir contidos por parte de aqueles que ou ben non poidan facer fronte á inversión que unha licenza supón ou ben non lles fose outorgada.

A medida que o acceso a Internet se fai máis cotián, a emisión por streaming eríxese coma solución aos problemas da radio en FM. A través deste medio, unha emisora pode emitir contidos sen necesidade de licenzas, acadando, ademais, unha cobertura global de xeito centralizado, sen necesidade de emitir mediante cadeas de radio enlace, como é costume nas grandes emisoras en FM do Estado Español, coa inversión en infraestruturas que tal cousa require.

Internet permite non só a emisión en directo mediante streaming, senón tamén o acceso a contidos baixo demanda co nacemento do podcasting a mediados da década dos 2000\cite{guardian}. A aparición de ferramentas de uso diario que permiten un acceso fácil a estes contidos está a afectar ao comportamento dos usuarios: Actualmente, un 7.5\% dos ouvintes escoitan a radio por Internet; aínda lonxe dos ouvintes de FM, porén máis do dobre dos de Onda Media\cite{EGM}.

\begin{figure}[H]
	\centering
	\includegraphics[scale=0.4,keepaspectratio=true]{./images/tabla_internet_radio.png}
  	\caption{Ouvintes de radio promedio diario do ano 2017 en España.}
	\label{fig:table_egm}
\end{figure}

\section{Motivación}

Ao entender que os suxeitos que atopan dificultades para emitir por FM son a miúdo medios do chamado terceiro sector. É dicir: entidades pequenas, comunitarias, sen ánimo de lucro, independentes e con finalidade maioritariamente cultural e social\cite{fesp}. O impacto positivo na pluralidade informativa e no desenvolvemento da sociedade civil destes medios é explicitamente recoñecido pola UNESCO\cite{unesco}. 

Se ben a emisión por streaming e o podcast se perfilan coma a alternativa nun futuro inmediato, tamén veñen acompañados de certa aura de incerteza. Hai que ter en conta que os intereses deste tipo de emisoras adoitan ser de ámbito local. Os eventos que unha radio comunitaria ten capacidade de cubrir rara vez son internacionais e os patrocinios aos que adoitan ter acceso son PEME's da propia localidade na que se atope a emisora. 

Deste xeito, a vantanxe de de globalidade que ofrece Internet acaba por non ser tal mentres que, pola contra, a súa visibilidade si queda diluída pola numerosa oferta existente, esta si, a nivel global.



\section{Obxectivos}
\label{obxectivos}

Este proxecto pretende servir de axuda aos colectivos do terceiro sector da comunicación, favorecendo a colaboración entre eles e achegándolles as ferramentas necesarias para acadar unha maior presenza na rede.

O produto resultante consistirá nun portal web onde os usuarios poderán acceder a un catálogo de programas. O mencionado catálogo estará constituído por arquivos de son enlazados por outros usuarios que desexen compartilos desde o seu propio almacenamento.

Os obxectivos a acadar son os seguintes:

\begin{itemize}
\item Facilitar desde un portal web o acceso a ficheiros de audio mediante streaming por Internet e descarga directa desde servidores alleos ao servizo.
\item Permitir a distintos usuarios a publicación, manual ou automatizada, do seu contido no sitio web de xeito organizado por programas e categorías.
\item Ofrecer aos visitantes ferramentas de procura de contidos e a opción de subscribirse aos diferentes programas ou canais.
\item Permitir a colaboración entre usuarios na xestión de contidos publicados.
\end{itemize}







  \include{estado_da_arte}
  \chapter[Tecnoloxía e ferramentas empregadas]{
  \label{chp:tecnologia}
  Tecnoloxía e ferramentas empregadas
}
\minitoc
\newpage

Neste capítulo, presentanse as linguaxes, ferramentas e frameworks utilizados no desenvolvemento deste proxecto. Á hora de elixir o uso destas ferramentas valorouse a súa natureza de código aberto xa que se pretende que o resultado dispoña dunha licenza de software libre compatible coa definición da Free Software Foundation. 

Tamén se valoraron, dado que nos capítulos anteriores se insistiu na importancia dos dispositivos móbiles, no consumo da radio a través de Internet, as posibilidades de ditas ferramentas de ofrecer un bo resultado en distintos dispositivos e distintas plataformas software. 


\begin{figure}[H]
	\centering
	\includegraphics[scale=0.55,keepaspectratio=true]{./images/diagrama_tech.png}
	\caption{Diagrama de interacción de tecnoloxías.}
	\label{fig:img_diagrama_tech}
\end{figure}


\section{Linguaxes}

\subsection{Python}
\label{python}
Python é unha linguaxe de propósito xeral de alto nivel. Trátase dunha linguaxe interpretada polo que é necesario 
ter un intérprete para executar o código. Ao ser o intérprete unha capa intermedia de software entre o programa
e o sistema, Python é unha linguaxe doadamente portable entre dispositivos de distinta natureza.

A súa sintaxe baseada na indentación está pensada para favorecer a comprensión entre distintos desenvolvedores e 
facilitar o mantemento do código. 

O seu sistema de tipado implícito e a súa riqueza de bibliotecas para executar comandos de consola de xeito programático
fan que sexa moitas veces descrito coma \say{unha linguaxe de scripting orientada a obxectos}\cite{python1}, o cal serve coma 
mostra da súa flexibilidade, un dos motivos polo que foi elixida para este proxecto.

\subsubsection{Anaconda}

Anaconda é unha distribución de Python, libre e de código aberto (licenza BSD), utilizada normalmente para análise estatística e machine learning. Inclúe, ademais dunha completa instalación do intérprete de Python, un rico conxunto de librarías de uso común e o xestor de paquetes conda, sendo esas dúas últimas melloras o motivo primordial polo que foi a elección para este proxecto. Conda utiliza os paquetes da comunidade de Conda Forge \cite{anaconda1}

Anaconda é responsabilidade de Anaconda Incorporated, antigo Continuum Analytics.

A versión utilizada foi a Anaconda 4.4.0 de 64 bits para Python 3, a máis nova no momento de comezar o traballo. Inclúe o intérprete para Python versión 3.6.1.


\subsection{HTML}

HTML (abreviación de HyperText Markup Language) é a linguaxe estándar para a creación da estrutura das páxinas web. Os 
elementos presentes decláranse mediante bloques representados por etiquetas(tags), de aí que reciba o nome de
\say{markup language} (linguaxe de marcado). Estas etiquetas son interpretadas polos navegadores para renderizar os 
contidos\cite{html1}. 

A estrutura declarada no código HTML pode ser representada pola interface de DOM (Document Object Model), como se amosa na figura \ref{fig:img_DOM}. O DOM é unha árbore creada polo navegador ao cargar a páxina e é moi útil para acceder aos elementos da páxina de xeito programático, como veremos máis adiante.

\begin{figure}[h]
	\centering
	\includegraphics[scale=0.7,keepaspectratio=true]{./images/DOM.png}
	\caption{Exemplo de árbore HTML DOM\protect\cite{dom}}
	\label{fig:img_DOM}
\end{figure}

Dada a natureza multimedia da web realizada, utilizouse HTML5. Esta quinta revisión do estándar inclúe novas etiquetas 
para o tratamento das imaxes e dos contidos de audio e vídeo sen necesidade de utilizar outras tecnoloxías complementarias, 
simplificando así o desenvolvemento e o mantemento posterior do portal. Desde decembro do pasado ano 2017, a versión 5.2 é a última estable recomendada polo World Wide Web Consortium(W3C)\cite{html_w3c}.

Ao ser o uso de HTML un estándar tan lonxevo, consolidado e popular, non se consideraron alternativas para este proxecto.


\subsection{CSS}

CSS (Abreviatura de Cascading Style Sheets) ou \say{Follas de estilo} é unha linguaxe empregada para definir a estética dos elementos definidos anteriormente no código HTML: A súa cor, fonte, posición... Ao separar o estilo da estrutura, favorécese a reutilización de código xa que un mesmo ficheiro de CSS pode ser utilizado en diversas páxinas ao mesmo tempo. CSS permite tamén adaptar o contido das páxinas a dispositivos de distinto tamaño\cite{css_w3c}.

\subsubsection{CSS Grid}
\label{cssgrid}

O módulo de \say{Grid} úsase para definir un deseño de interface gráfica consistente nunha cuadrícula de dúas dimensións en CSS. Nun modelo deste tipo, declárase un conxunto de elementos no HTML coma pertencentes a un \say{grid container}(contedor da grella ou da cuadrícula) e, á súa vez, zonas fillas que tomarán posición dentro dese contedor dependendo das características coas que este último fose definido na folla de estilos.

Unha cela nun contedor pode, á súa vez, consistir nun contedor en sí mesma, permitindo así deseños asimétricos. O tamaño das celas pode ser fixo, relativo á páxina ou automático dependendo do contido da cela. Por todo o dito, este módulo proporciona un nivel de flexibilidade superior ao que poderíamos obter utilizando outras alternativas populares coma CSS Flexbox. 

O nivel primeiro de CSS Grid non acadou aínda, na data de entrega desta memoria, o status de \say{recomendación} da  World Wide Web Consortium(W3C)\cite{css_grid_w3c}, porén, xa é soportado polas últimas versións dos navegadores máis populares\cite{css_grid_w3c2} polo que non se considerou un risco utilizalo neste proxecto.

\subsection{JavaScript}

JavaScript é unha linguaxe de scripting interpretada e de alto nivel. Utilízase principalmente (e tamén neste proxecto) coma unha ferramenta para mellorar a interacción coa interface web ao permitir executar código orientado a eventos no lado do cliente, aforrando recargas da páxina e incluso accesos innecesarios á base de datos xa que permite o uso do disco local mediante cookies. 

A pesares do seu nome, non garda relación algunha coa linguxe Java e serven a propósitos claramente distintos. O núcleo desta linguaxe está regulado polo estándar ECMAScript\textregistered, na súa 8\textsuperscript{a} versión\cite{ecma} no momento de entregar esta memoria. 

Como se comentou no apartado \ref{python}, que sexa interpretada implica a necesidade dun intérprete para executar o código. Ese intérprete, comunmente chamado JavaScript Engine preséntase embebido nos navegadores web. Non obstante, non existe unha única implementación senón que distintos navegadores presentan distintas versións. Para o desenvolvemento deste proxecto utilizáronse os navegadores Mozilla Firefox 57.0.1 e Google Chrome 66.0, cuxos motores de JavaScript son SpiderMonkey e V8 respectivamente\cite{javascript1}, ambos os dous libres e de código aberto.

Pese a que o seu uso nos navegadores continúa a ser a principal razón de ser desta linguaxe, tamén se utiliza a día de hoxe noutro tipo de produtos coma Node.js(para correr JavaScript no lado do servidor) ou Apache Couch DB (Para o manexo de bases de datos na nube)\cite{javascript2} 

Ademais do código en JavaScript puro, este proxecto tamén fai uso nalgunhas partes do código de funcións de jQuery.

\subsubsection{jQuery}
\label{jquery}
JQuery é unha biblioteca de JavaScript creada co fin de simplificar o \say{scripting} no lado do cliente. Trátase dunha ferramenta libre, publicada baixo licenza MIT e é mantida pola comunidade jQuery Team. No 2015, xa era empregada polo 63\% do Top Million Websites\cite{jquery}, incluíndo sitios populares coma Netflix, Amazon ou Microsoft. O seu uso segue a ser moi xeralizado a día de hoxe.  

Cunha API funcional e válida en gran parte dos navegadores web, evita moitos dos problemas de incompatibilidade que aparecen ao utilizar JavaScript puro e o código resultante adoita ser moito máis conciso, mellorando a súa comprensibilidade e o seu mantemento. Isto débese en grande medida pola súa expresión \say{selector} que facilita o acceso aos distintos elementos da árbore do HTML DOM (ver figura \ref{fig:img_DOM}). 

A versión de jQuery utilizada foi a 2.1.4 por ser a máis actual incluída na biblioteca django-static-jquery-2.1.4\cite{dj-jquery} do framework utilizado (ver sección \ref{django}) ao comezo do desenvolvemento.



\subsection{SQL}

SQL é a linguaxe utilizada para a manipulación dos datos dentro do eido das bases de datos relacionais. Nace coma un refinamento ou \say{secuela} (SQL é unha simplificación da verba inglesa \say{sequel}) da linguaxe SQUARE, que á súa vez consistía nunha simplificación da linguaxe DSI/Alpha proposta polo mesmo Edgar F. Codd no momento de presentar o modelo relacional de Bases de Datos. O primeiro estándar foi publicado no ano 1986 pola American National Standards Institute (ANSI)\cite{sql1}. Permite definir a estruturación dos datos, inserilos, eliminalos, editalos e, por suposto, consultalos.

Neste proxecto, o seu uso explícito correspondeuse so coas primeiras fases do traballo debido á utilización do framework Django coma se detalla no apartado \ref{django}. É necesario especificar tamén que, pese á existencia dun estándar, existen distintas versións desta linguaxe que a fan \say{non completamente portable} entre distintos sistemas de xestión de bases de datos\cite{sql2}. Neste proxecto utilizouse a variante correspondente a PostgreSQL.


\subsection{XML}

XML é a abreviatura de \say{eXtensible Markup Language} (Linguaxe de marcado extensible). Consiste nunha serie de normas que dividen un documento en distintas partes, asignándolle unha identidade a cada unha delas. A diferencia doutras linguaxes de marcado coma o HTML, non existe un conxunto de etiquetas válido. No XML queda á responsabilidade do autor do documento establecer as etiquetas necesarias dependendo do entorno no que ese documento vaia ser utilizado. Por suposto, existen unha serie de normas estruturais para que o etiquetado se poida considerar coma válido, pero desde o punto semántico, a natureza destas etiquetas é moi flexible. Isto fai que moitas veces se denomine o XML coma unha Meta-Linguaxe, é dicir, unha linguaxe para definir linguaxes\cite{xml1}.

A función de XML é definir a semántica e a estrutura dun documento, pero nunca o estilo. Ao igual que HTML, pódese asociar unha folla de estilos CSS.

Neste proxecto, XML é utilizado na súa modalidade RSS do xeito detallado no apartado \ref{rss}.

\subsubsection{RSS}
\label{rss}

\begin{figure}[h]
	\centering
	\includegraphics[scale=0.6,keepaspectratio=true]{./images/hlk_rss.png}
	\caption{Fragmento de ficheiro RSS real utilizado nas probas.}
	\label{fig:rss}
\end{figure}

RSS é un tipo de formato XML utilizado para a \say{sindicación} web, isto é, a emisión de contidos actualizados dunha páxina web a un número indefinido de subscritores. Utilízase para evitar a procura manual de novos contidos naquelas páxinas nas que a actualización é frecuente: Páxinas de novas, blogs, podcasts... 

O seu funcionamento consiste nun ficheiro RSS (comunmente chamado \say{feed}) accesible desde a web polos programas clientes de rss que posúan os subscritores. Ese feed mostra información resumida do contido publicado no sitio web ao que fai referencia. Cada vez que o contido se actualice, o feed actualizarase tamén e ese cambio será detectado polo cliente rss na súa seguinte comprobación mediante \say{polling}.
 
O nome procede do acrónimo de \say{Really Simple Syndication}(Sindicación Realmente Sinxela) e o estándar é mantido polo \say{RSS Advisory Board}. Na data de entrega desta memoria e desde o ano 2009, a última revisión é a 2.0.11 \cite{rss}

Neste proxecto, o interese céntrase na actualización dos podcasts. Existen alternativas a este formato de sindicación mais, dado o seu uso xeralizado entre os usuarios potenciais da aplicación web a desenvolver, non foron consideradas.


\subsection{JSON}

\begin{figure}[h]
	\centering
	\includegraphics[scale=0.5,keepaspectratio=true]{./images/json.png}
	\caption{Exemplo de sintaxe JSON\cite{json_wiki}}
	\label{fig:json}
\end{figure}

JSON é un estándar de sintáctico utilizado para o intercambio de datos lixeiros de texto entre distintas linguaxes. A pesares de ser a abreviatura de JavaScript Object Notation pola súa orixe inspirada nos tipos desta linguaxe\cite{json} o seu uso por parte doutras tecnoloxías é común. Esta linguaxe non define un sistema completo de intercambio de datos, mais si un marco sintáctico sobre o que definir un.

Baséase nunha xerarquía definida por chaves, corchetes, comas, dous puntos e parénteses no xeito descrito na figura \ref{fig:json}, onde se amosa un exemplo de codificación dos datos persoais dun home.

Neste proxecto, utilizouse de forma implícita ao ser a codificación utilizada polos obxectos de contexto de Django.

\subsection{LaTeX}
\label{latex}

LaTeX é un sistema de preparación de documentos para os que sexa necesaria unha tipografía de alta cualidade, habitualmente, documentos medios ou longos para publicacións científicas. A motivación de LaTeX é a máxima separación posible entre o contido e o estilo do documento, facendo depender o segundo de modelos preexistentes para que os autores podan concentrarse na creación do primeiro.

Este sistema non é un editor de texto. Consiste nun conxunto de macros e un programa procesador que interpreta os mesmos. É un software libre e de código aberto (publicado baixo licenza  LaTeX Project Public License\cite{latex}). Na actualidade, pódese escoller entre distintas  distribucións libres de LaTeX que inclúen unha ampla variedade de paquetes de macros adicionais. Para a confección desta memoria, utilizouse TexLive na súa versión de 2016 por ser a máis actual dispoñible nos repositorios do Sistema Operativo no momento de comezar a escritura.






\section{Django Framework}
\label{django}
O proxecto Django nace no ano 2005 coma un conxunto de ferramentas Python para o desenvolvemento web publicadas
baixo licencia BSD. A motivación dos seus creadores, Simon Wilson e Adrian Holovaty, naquel tempo programadores
nun medio periodístico, era a homoxeneización e a reutilización de código. Isto daba resposta á necesidade de reducir
o alto custo que supón manter código ad-hoc para cada novo artigo ou función, especialmente nunha páxina que
requira unha constante actualización de contidos coma a dun diario en liña. É por iso que se adoita utilizar o 
termo \say{newsroom schedule} cando se fala da rapidez de desenvolvemento que permite o framework\cite{django1}.  

Django ofrece un conxunto de bibliotecas que dan soporte ás mecánicas máis comúns do desenvolvemento
web:

\begin{itemize}
	
	\item Persistencia: Django permite crear a estrutura da base de datos sen necesidade de escribir SQL. Isto 
	conséguese mediante a superclase Model da librería \say{db} de Django. Ao declarar unha clase de Python coma
	extensión de \say{Model}, a biblioteca ocúpase automaticamente de manter a coherencia entre as súas instancias en 
	memoria e as táboas da base de datos. 
	
	\item Peticións web: A lectura e a interpretación destas peticións son levadas a cabo polo conxunto de bibliotecas
	de HTTP do framework, encapsulando as peticións e as respostas en obxectos para formar un estándar sinxelo e garantir
	a correcta formación das mesmas.
	
	\item Enrutamento e Validación: O framework ofrece un estándar para asignar as distintas vistas definidas no código
	ás URL's desexadas. Os posibles formularios presentes nesas vistas poden ser abstraídos en obxectos de Python 
	simplificando a validación dos datos introducidos polos usuarios. A utilización destes sistemas é non obstante, 
	opcional, podendo o programador optar por utilizar sinxelos formularios HTML no caso de que isto axilizase 
	o desenvolvemento.
	
	\item Sistema para embeber datos dinámicos no HTML: O chamado \say{template system} de Django marca uns procedementos
	sinxelos para acceder aos datos xerados no código. Tamén ofrece ferramentas para implementar certa lóxica
	na presentación do front end.
	
	\item Interface de administración nativa: Por defecto, este framework dispón dun panel de administración de acceso
	reservado aos superusuarios que evita os accesos directos á base de datos no caso de que un administrador necesite
	facer cambios manuais nos datos.
	
	\item Soporte por defecto para a autenticación e autorización: Django permite ao desenvolvedor invocar a clase 
	nativa Usuario e efectuar con ela accións de rexistro, autenticación e concesión/revogación de permisos sen máis
	programación.

\end{itemize}   


Para este proxecto utilizouse concretamente Django 1.11, a versión estable máis avanzada no tempo en que se comezou o 
traballo. Django sufriu unha actualización importante desde entón que rematou coa publicación de Django 2.0 en Decembro do pasado ano 2017. Estas dúas versións non son totalmente compatibles de modo que unha actualización implicaría cambios no 
código. Tendo en conta que a Django Software Foundation, responsable do mantemento do framework, non prevee retirar o soporte á versión utilizada nun futuro próximo e que a versión de Python utilizada é soportada tamén por Django 2\cite{django2}, considerouse aceptable o grao de actualidade da tecnoloxía utilizada.    

Ademais do mencionado anteriormente, confiouse en Django para este proxecto pola súa madurez e pola súa popularidade, sendo  utilizado en sitios coma Instagram, Disqus, Pinterest ou Open Knowledge Foundation.


\section{Celery}

Celery é unha cola de traballos asíncrona baseada na mensaxería distribuída (distributed message passing). Utilízase para distribuír traballos entre máquinas ou fíos. Celery levanta un conxunto de procesos chamados \say{workers} nos que se executarán de xeito concorrente os diferentes traballos ou \say{tasks}. Eses workers consultan constantemente a cola de traballos para executar os traballos que podan estar pendentes\cite{celery}.

Neste proxecto, utilizouse Celery para controlar a execución dos demos que corren no lado do servidor. Eses dous demos son: O encargado de actualizar a información do programa e os episodios e o encargado de calcular a popularidade dos programas. Utilizouse a versión de Celery 4.1, a máis actual no repositorio de Anaconda no momento de comezar o traballo.

Django ofrece bibliotecas para traballar con Celery de xeito máis sinxelo, por exemplo, permitíndolle gardar información na base de datos do proxectos e engadindo ferramentas de administración dos procesos de Celery ao menú de administración de Django. Utilizáronse as bibliotecas django\_celery\_results 1.0.1 e django-celery-beat 1.1.1

Como xa se mencionou, Celery baséase na mensaxería, mais non inclúe un sistema de seu senón que lle hai que proporcionar un. Neste caso, optouse pola utilización de RabbitMQ (ver sección \ref{rabbit}), un dos recomendados na documentación de Celery.

\section{RabbitMQ}
\label{rabbit}

RabbitMQ é un software de \say{message passing} tipo \say{broker} mensaxes. Isto é, un sistema onde diferentes aplicacións se conectan ao broker coa fin de enviar a ou lelos deste, funcionando o dito broker coma unha cola. As mensaxes enviadas á cola por un aplicativo permanecerán aí até que sexan lidos por outro aplicativo. Unha mensaxe pode ser calquera cousa, desde meta información dos procesos até texto plano\cite{rabbitmq}.

Neste proxecto utilizouse a versión 3.6.10-1 por seres esta a máis actual dispoñible nos repositorios do Sistema Operativo no momento de comezar o traballo.


\section{Bootstrap}

Bootstrap é un framework libre e de código aberto (publicado baixo licenza MIT) para a construción de interfaces web. O Framework ofrece unha serie de modelos ou \say{templates}, cada unha cunha estrutura HTML, declaracións en CSS e, nalgúns casos, extensións JavaScript.

Neste proxecto, utilizouse a través da biblioteca de django-bootstrap4 para facilitar o uso dos elementos da versión 4 de Bootstrap desde os templates do framework de Django.


\section{Ajax}

Ajax é a abreviación de \say{Asynchronous JavaScript and XML}. Trátase dunha técnica combinada desas tecnoloxías, aínda que a miúdo substituíndo XML por JSON\cite{ajax}, utilizado para manter unha comunicación entre o lado cliente e o lado servidor nunha páxina web sen necesidade dunha recarga completa. 

Utilizando Ajax pódese facer unha petición ao servidor e recibir información de xeito asíncrono mediante JSON ou XML para, mediante a manipulación do HTML DOM con JavaScript, actualizar só partes da páxina amosada ao usuario. Utilízase para diminuír o tráfico entre os dous lados, facendo a navegación máis áxil e mellorando a interactividade.

Nesta aplicación utilizouse a través dos métodos definidos por jQuery (ver apartado \ref{jquery})

\section{Apache HTTP server}

Apache é un proxecto colaborativo para o desenvolvemento dun servidor HTTP robusto, con cualidade suficiente para o seu uso comercial, libre e de código aberto (publicado baixo licenza Apache, versión 2.0\cite{apache}). Apache implementa o protocolo HTTP/1.1 (RFC2616) xunto con varias funcionalidades frecuentemente requiridas pola web coma a personalización das mensaxes de erro (incluso mensaxes que conteñan de CGI scripts), a posibilidade de redirección e declaración de \say{alias} para as URLs de xeito ilimitado ou tamén soporte para hosts virtuais.

A pesar de que o seu dominio coma servidor máis utilizado decaeu nos últimos anos, segue a ser a opción libre máis popular, estando presente nun 36.34\% dos sitios web pertencentes ao \say{Top million busiest sites} segundo datos do mes de abril do presente ano 2018 (ver figura \ref{fig:img_netcraft}). Na actualidade é utilizado en sitios ben coñecidos coma o a páxina de novas da BBC, o sistema de pago PayPal ou a tenda de videoxogos en liña Steam\cite{netcraft}. Esta popularidade é a razón principal pola que se utilizou este software no proxecto.

\begin{figure}[h]
	\centering
	\includegraphics[scale=0.6,keepaspectratio=true]{./images/apache_graph.png}
	\caption{Cota de mercado dos servidores no Top million busiest sites (Netcraft, abril 2018)}
	\label{fig:img_netcraft}
\end{figure}

A versión utilizada foi Apache 2.4.27 por ser a máis actual no momento de comezar o traballo. Foi necesaria tamén a instalación das seguintes extensións:

\begin{itemize}
	\item Apache Portable Runtime (APR): Biblioteca que inclúe unha serie de API's para garantir o funcionamento de Apache sexa cal sexa a plataforma na que se execute. Utilizouse a versión 1.6.2 e a 1.6.0 da biblioteca complementaria APR-util. 
	
	\item mod\_wsgi: Módulo de Apache que poporciona unha WSGI (Web Server Gateway Interface), necesaria para albergar aplicativos web baseados en Python. Utilizouse a versión 4.5.17.
\end{itemize}


\section{PostgreSQL}

PostgreSQL é un sistema de xestión de bases de datos (SXBD) relacionais libre e de código aberto (Licenza PostgreSQL, permisiva) desenvolvido polo PostgreSQL Global Development Group. A linguaxe SQL soportada pretende ser o máis fiel posible ao estándar e as súas operacións cumpren coas regras ACID (Acrónimo inglés para Atomicidade, Consistencia, Illamento e Durabilidade)\cite{postgres1}. A súa orixe remóntase a 1996, polo que existe unha grande dispoñibilidade de recursos de documentación. Ese feito unido á certa experiencia persoal no seu uso xa antes de comezar o proxecto foron valorados á hora de elixir este sistema.

A versión utilizada é a 9.6-3 por ser a máis actual no momento de comezar o traballo. Para facer posible o acceso mediante Python, utilizouse o driver psycopg2 na súa versión 2.7.1


\section{Ferramentas de desenvolvemento}

\subsection{Eclipse}
\label{eclipse}

Eclipse é un IDE(Integrated Development Environment ou Entorno de Desenvolvemento Integrado), isto é, un aplicativo consistente nunha serie de ferramentas de desenvolvemento de software dispoñibles arredor dun editor de texto tales coma, por exemplo, unha consola de saída estándar e ferramentas para compilar e depurar código desde a interface proporcionada.

É principalmente empregado nos proxectos de Java, C e C++, porén, dada a súa popularidade, existen extensións dispoñibles no seu propio \say{marketpace} que nos permiten utilizalo para outras linguaxes. No caso deste proxecto, ese plugin é PyDev, que proporciona as ferramentas necesarias para o desenvolvemento non só de proxectos de Python en xeral senón especificamente de Django.

Outra extensión utilizada foi EGit na súa versión 4.8. Esta serve para manexar o control de versións  desde a interface gráfica de Eclipse. Egit utiliza JGit, unha implementación puramente Java do sistema Git (ver apartado \ref{git}).

Optouse pola versión de Eclipse máis nova no momento de comezar o traballo pese a que aínda non estaba dispoñible daquela nos repositorios do sistema operativo do equipo de desenvolvemento: Eclipse Oxygen 4.7.0. A versión de PyDev utilizada foi a 5.9.2. Valorouse PyCharm, un IDE específico para Python, coma alternativa; mais a versión de comunidade non tiña soporte para desenvolvemento específico de Django\cite{pycharm}.
 


\subsection{Git}
\label{git}

Git é un sistema de control de versións libre e de código aberto (licenza GPL v2). Un sistema de control de versións utilízase para gardar os diferentes estados (versións) do código nas distintas fases de desenvolvemento. Permite a creación de ramas para a división do traballo e de etiquetas para marcar certos hitos no proceso. Estas versións pódense gardar no equipo local ou nun repositorio remoto. No caso de Git, mantéñense copias tanto no repositorio coma nos equipos dos programadores, polo cal adoita cualificarse coma un sistema de control de versións distribuído\cite{git}.

Como se mencionou no apartado \ref{eclipse}, utilizouse sobre todo na súa implementación de EGit, mais as veces recorreuse ao paquete de sistema para realizar operacións sen necesidade de acceder a Eclipse. Neses casos, utilizouse a versión 2.9.5.

Existen outros sistemas coma, por exemplo, Apache Subversion, pero elixiuse Git pola posibilidade de publicar o código en GitHub

\subsubsection{GitHub}  

GitHub é un servizo de hosting web para repositorios de control de versións. É moi popular para o desenvolvemento colaborativo de proxectos de código aberto. Neste proxecto, utilizouse ata o momento coma alternativa gratuíta para obter un repositorio en liña pero, idealmente, tamén será utilizado para colaborar con outros desenvolvedores no futuro. 


\subsection{Dia}

Dia é un editor de Diagramas libre e de código aberto(Licenza GPL). Utilizáronse neste proxecto as súas extensións de UML para o diagrama de clases do sistema e a de Entidade Relación para o esquema de deseño da base de datos. A versión utilizada foi a 0.97.3.

Consideráronse outros editores coma, por exemplo, Umbrello, pero a elección foi DIA por criterios de estética e comodidade de uso debido á súa interface sinxela.

\subsection{Fedora}

Fedora é un Sistema Operativo que utiliza o kernel de Linux. É desenvolvido pola comunidade do Fedora Project, patrocinada desde os seus inicios por Red Hat Incorporated. Se ben as súas distintas compoñentes non están publicadas baixo a mesma licenza, estas si se enmarcan na definición de \say{Licenza de Software Libre} da FSF (Free Software Foundation) a excepción dalgúns ficheiros de firmware que, por especificación dos fabricantes de hardware, teñan que estar presentes unicamente coma arquivos binarios. Nese último caso, han de cumprimentar unha serie de requirimentos coma estar libres de pago polo seu dereito de uso\cite{fedora}. Podemos entender este, polo tanto, coma un Sistema Operativo libre.

Todo o desenvolvemento deste proxecto así coma a escritura desta memoria realizouse sobre Fedora 25 (64 bits), utilizando Xfce 4.12 coma entorno de escritorio.


\subsection{TeXstudio}

TeXstudio é un editor de documentos LaTex libre e de código aberto (Licenza GPL v2\cite{texstudio}). Proporciona, entre outras funcionalidades, un editor gráfico con soporte de marcado ortográfico para distintos idiomas, un visor de ficheiros PDF integrado e un corrector de referencias. Para este proxecto, utilizouse a versión 2.12.6 co paquete ortográfico de lingua galega de Libre Office v13-10.

Trátase unicamente dun editor, non inclúe un procesador de macros de LaTex. Utilizouse para isto a distribución TeX Live 2016 (ver sección \ref{latex}) e, para a exportación da memoria a formato PDF, pdfTex na súa versión 3.14159265 (a incluída no paquete).

\subsection{Mozilla Firefox}

Firefox é un navegador web libre e de código aberto (Licenza Mozilla Public License v2\cite{firefox}). Soporta unha grande variedade de estándares e está dispoñible en gran variedade de sistemas operativos, incluíndo os dos dispositivos móbiles máis populares: iOS e Android. Aínda que a súa popularidade foi en retroceso nos últimos anos, segue a ser un dos navegadores máis populares e o máis utilizado entre as alternativas de software libre (ver figura \ref{fig:firefox}).

Valorouse, á hora de decidir utilizalo, a súa importancia no mundo do software libre, a súa dispoñibilidade de serie no Sistema Operativo utilizado e as súas ferramentas de desenvolvemento, incluído o seu modo de deseño \say{responsive}. Utilizouse a versión 57.0.1 de 64 bits, a máis actual dispoñible nos repositorios de Fedora 25.  

\begin{figure}[h]
	\centering
	\includegraphics[scale=0.5,keepaspectratio=true]{./images/firefox.png}
	\caption{Cota de mercado dos navegadores web. (W3Counter, maio 2018)\cite{firefox2}}
	\label{fig:firefox}
\end{figure}



\subsection{GIMP}

GIMP é a abreviatura de GNU Image Manipulation Program (Programa de manipulación de imaxes de GNU). É un software de edición de imaxes libre e de código aberto (Licenza GPL v3\cite{gimp}). Neste proxecto utilizouse para facer postprocesados sinxelos das imaxes incluídas na memoria. A versión utilizada foi a 2.8.22 por ser a máis actual nos repositorios do Sistema Operativo no momento de comezar a escritura da memoria.

\subsection{Projectlibre}

Projectlibre é unha ferramenta de xestión e dirección de proxectos, libre e de código aberto (licenza CPAL\cite{projectlibre}). Naceu coma alternativa libre a Microsoft Project, sendo compatible con este. Ofrece diversas funcionalidades coma os diagramas de Gantt, histogramas de recursos ou esquemas RBS (Resource Breakdown Structure) 

  \chapter[Metodoloxía]{
  \label{chp:metodoloxia}
  Metodoloxía
}
\minitoc
\newpage

A metodoloxía de traballo baseouse nos principios das \textbf{metodoloxías áxiles} aplicando, concretamente, aquelas prácticas e técnicas de \textbf{\say{Extreme Programming}} (XP) adaptables ao traballo individual. Destacarase, pola súa especificidade, o uso da técnica de \say{Test Driven Development} (TDD) nalgunhas partes do desenvolvemento.

\section{Principios das metodoloxías áxiles}

Considéranse áxiles aquelas metodoloxías motivadas por unha serie de valores e que respectan unha serie principios, ambos recollidos no \say{manifesto áxil} do ano 2001 \cite{axiles}. Eses valores son:

\begin{itemize}
	\item \textbf{Os individuos e súa interacción} valórase máis que as ferramentas e os procesos.
	\item \textbf{O correcto funcionamento do software} valórase máis que a documentación extensiva.
	\item \textbf{A colaboración co cliente} valórase máis que a negociación contractual.
	\item \textbf{A resposta ante o cambio} valórase máis que o seguimento dun plan.
\end{itemize}

Que se traducen nos chamados \textbf{\say{Doce principios do Software Áxil}}:

\begin{enumerate}
\item A nosa maior prioridade é satisfacer ao cliente
mediante a entrega temperá e continua de software
con valor.

\item Aceptamos que os requirimentos cambien, incluso en etapas 
tardías do desenvolvemento. Os procesos Áxiles aproveitan
o cambio para proporcionar vantaxe competitiva ao 
cliente.

\item Entregamos software funcional frecuentemente, entre dúas
semanas e dous meses, con preferencia ao período de 
tempo máis corto posible.

\item Os responsables de negocio e os desenvolvedores
traballamos xuntos de forma cotiá durante todo
o proxecto.

\item Os proxectos desenvólvense en torno a individuos 
motivados. Hai que darlles o entorno e o apoio que 
necesitan, e confiarlles a execución do traballo. 

\item O método máis eficiente e efectivo de comunicar 
información ao equipo de desenvolvemento e entre os seus 
membros é a conversación cara a cara.

\item O software funcionando é a medida principal de 
progreso.

\item Os procesos Áxiles promoven o desenvolvemento 
sostible. Los promotores, desenvolvedores e usuarios
debemos ser capaces de manter un ritmo constante 
de forma indefinida.

\item A atención continua á excelencia técnica e ao 
bo deseño mellora a Axilidade.

\item A simplicidade, a arte de maximizar a cantidade de
traballo non realizado, é esencial.

\item As mellores arquitecturas, requirimentos e deseños
emerxen de equipos auto-organizados.

\item A intervalos regulares o equipo reflexiona sobre
como ser máis efectivo para a continuación axustar e
perfeccionar o seu comportamento en consecuencia.
\end{enumerate}


\section{Extreme Programming}

Extreme Programming (XP), ou Programación Extrema, é un estilo de desenvolvemento de software que promove unha serie de valores de cara ao traballo en equipo e unha serie de técnicas que supoñen unha simplificación e unha alternativa flexible en contraposición a outras metodoloxías máis tradicionais coma, por exemplo, o \say{desenvolvemento en fervenza}.

Eses valores enuméranse coma\cite{xp}: 

\begin{itemize}
	\item \textbf{Comunicación:} Tanto entre programadores coma co cliente.
	\item \textbf{Simplicidade:} Que o código sexa lexible e a documentación concisa.
	\item \textbf{Retroalimentación:} Realizar iteracións curtas para obter opinións rápidas sobre os progresos.
	\item \textbf{Coraxe:} Aceptar que a aparición de novos requirimentos é natural e poden redefinir o deseño.
	\item \textbf{Respecto:} Estrutura horizontal do equipo de desenvolvemento.
\end{itemize}


A posta en práctica dos anteriores valores lévanos a unha metodoloxía de traballo baseada no contacto continuo co cliente para poder obter correccións e ideas novas (retroalimentación). Isto acádase mediante ciclos curtos de desenvolvemento, ao final dos cales volveremos reunirnos co usuario obxectivo e o ciclo volve a comezar. 

Referímonos a estes ciclos coma \textbf{iteracións} e a esas \say{ideas novas} coma \textbf{historias de usuario}. Poden ser tomadas coma unha planificación do traballo a curto prazo a entender non coma unha improvisación, senón coma unha peza dun plan global suxeito á evolución do proxecto.

Debido á aceptación de que os sucesivos cambios no deseño son inevitables (coraxe) e que a refactorización de código é algo necesario para asegurar a súa cualidade (simplicidade), faise necesaria a automatización de probas e a utilización dun control de versións.

A metodoloxía XP ten unha serie de características aplicables ao traballo en equipo que non se levaron a cabo neste proxecto: A programación por parellas, a revisión de código entre desenvolvedores ou a división do traballo en función da especialización do persoal (Testers, deseñadores de interacción, directores do proxecto, programadores...)

\subsubsection{Aporte da metodoloxía ao proxecto}

Ao ser este un Proxecto de Final de Carreira, as reunións co cliente substituíronse por \textbf{reunións co director do proxecto de periodicidade quincenal}. Nelas, revisábanse os progreso e xurdían novas historias de usuario (ou se actualizaban as vellas). Grazas a este modo de traballo conseguíronse identificar varios \textbf{requirimentos non valorados a priori} coma, por exemplo, a necesidade de establecer roles de usuario. 

Para afondar individualmente en cada iteración, ver o capítulo \ref{chp:plan} sobre a planificación.


\section{Test Driven Development}

Test Driven Development (TDD) é unha técnica utilizada nas metodoloxías áxiles de desenvolvemento que segue a filosofía \textbf{probar primeiro}. Esta baséase na conversión dos requirimentos en probas antes da propia existencia de código que as avale. Isto obriga ao desenvolvedor a pensar nos requirimentos con detemento, identificar de antemán os posibles fallos e aumentar o alcance da validación de datos. Unha vez o test está definido, pasarase a implementación da funcionalidade posta a proba. De pasar a proba, repítese co seguinte requirimento. Este ciclo pode verse no esquema da figura \ref{fig:tdd}.

\begin{figure}[h]
	\centering
	\includegraphics[scale=0.55,keepaspectratio=true]{./images/TDD.png}
	\caption{Diagrama do ciclo de desenvolvemento coa metodoloxía TDD.}
	\label{fig:tdd}
\end{figure}

Existen distintas estratexias á hora de seguir o citado ciclo \cite{tdd} que se poden utilizar en función das características do requirimento:

\begin{itemize}
	\item \textbf{Fake it ('til you make it):} Ou \say{falséao (ata que o fagas)}. Baséase en escribir primeiro un método falso que devolva un valor constante que satisfaga o test. A partir de aí, ilo modificando aos poucos sempre vixiando que o test sexa positivo até que o método estea completo. Esta práctica é moi complexa de utilizar se os métodos son grandes, polo que pode ser unha boa forma de forzar a división, algo desexable para a cualidade do código.
	
	\item \textbf{Triangulación:} Consiste en facer as mesmas comprobación 2 ou máis veces con parámetros distintos dos que esperemos unha saída distinta coa fin de abstraer a función a implementar a raíz dos resultados esperados. Esta implementación é útil para evitar os parámetros superfluos que poidan aparecer nas funcións complexas.
	
	\item \textbf{Implementación Obvia:} Para as funcionalidades sinxelas, non paga a pena utilizar estratexias de abstracción. Podemos simplemente implementar o método e agardar a que o test non falle. Queda a discreción do desenvolvedor definir que funcións son sinxelas e complexas.
	
	\item \textbf{Un a Moitos:} No caso daqueles métodos que actúan sobre unha pluralidade de obxectos, esta estratexia avoga por dividir o requirimento en dous: Habería que validar primeiro o un método que actuaría sobre unicamente un obxecto e, a continuación, validar que se execute sobre dita colección.  

\end{itemize}


Neste proxecto, seguiuse maioritariamente a terceira estratexia da anterior lista por resultar máis intuitiva para o traballo a realizar.

\subsubsection{Aporte da técnica ao proxecto}

Considerouse a utilización de TDD durante a \textbf{implementación do modelo de datos} (ver primeiras iteracións no capítulo \ref{chp:plan}), ao predicirse que sería un código moi sensible aos cambios podería dar pé a regresións. Tamén axudou ao análise de requirimentos ao forzarnos a pensar nos resultados posibles de cada método de antemán.

O uso de TDD forzou a temperá implementación de probas automatizadas, que se poden ver detalladas no capítulo \ref{chp:test} sobre as probas do sistema.



  \chapter[Planificación]{
  \label{chp:plan}
  Planificación
}
\minitoc
\newpage

Neste capítulo explicarase a planificación do traballo realizado e a avaliación de custes.  

\section{Iteracións}

Como se explicou no capítulo \ref{chp:metodoloxia}, o desenvolvemento executouse de  de forma iterativa e incremental onde unha iteración son os obxectivos a cumprir entre dúas reunións co cliente. Ao ser isto un proxecto de final de carreira, contarase o tempo entre as sesións de revisión de progresos entre o director do proxecto e o alumno.

\subsection{Reunión inicial}

Produciuse un encontro con membros de Cuac FM (a radio comunitaria da Coruña) e a URCM onde se entrou en contacto cos usuarios finais do proxecto a realizar. A URCM (Unión de Radios Comunitarias de Madrid) está composta por unha serie de emisoras independentes e con programas de seu; todos eles listados nunha sección da web (ver figura \ref{fig:urcm}) que á súa vez da acceso ao directorio dos ficheiros de audio (aos que a partir de agora nos referiremos coma \say{audios}) publicados por ditas emisoras. Esta sección da web, pese a súas limitacións, leva funcionando uns anos e resultou ser moi positiva para a redifusión dos programas por distintos colectivos.

\begin{figure}[h]
	\centering
	\includegraphics[scale=0.55,keepaspectratio=true]{./images/urcm.png}
	\caption{Sección de audios da web da URCM que inspira o proxecto.}
	\label{fig:urcm}
\end{figure}

Inspirados nesta idea, propúxose crear unha ferramenta semellante, esta vez para a ReMC (Red de medios comunitarios), unha federación de medios comunitarios do Estado Español. Definíronse uns requirimentos iniciais, máis centrados naquel entón na
redifusión e a organización que na escoita.

\subsubsection{Requirimentos primitivos}

A seguinte lista é unha transcrición das notas tomadas durante esa primeira reunión:

\begin{itemize}
	\item Os usuarios dos sistema son as propias emisoras.
	\item As emisoras teñen que poder engadir audios.
	\item As emisoras poden acceder aos audios das outras.
	\item As emisoras teñen que poder saber quen está a emitir os seus programas.
\end{itemize}



  \chapter[Análise]{
  \label{chp:analise}
  Análise
}
\minitoc
\newpage

Neste capítulo descríbese o funcionamento ideal da aplicación. Exporanse tamén os requirimentos funcionais e non funcionais identificados ao abordar os obxectivos do proxecto. 

\section{Descrición do funcionamento}

O sistema consiste nunha páxina web con apartados adicados, por un lado, ao que chamamos actividades de consumo, e por outro ás actividades de produción.

\subsection{Actividades de consumo de contidos}

Son aquelas \textbf{levadas a cabo polos ouvintes} das emisoras e os programas presentes no sistema. Un usuario, no seu papel de \say{consumidor}, pode, mediante a interface web, acceder ao catálogo de programas, xa sexa mediante as ferramentas de busca proporcionadas ou a través das recomendacións que a propia páxina amosa en portada.

Dentro da páxina correspondente a unha emisora, o usuario atopará un reprodutor HTML5 desde o que \textbf{escoitar a súa emisión en directo} mediante streaming. O usuario terá opción de \textbf{facerse \say{seguidor}} da emisora para acceder a ela de xeito máis rápido desde a páxina principal.

Tamén obterá acceso á lista de programas emitidos por dita emisora co seu horario de emisión. Na páxina de cada programa verá a lista completa de episodios dispoñibles e terá a opción de \textbf{subscribirse} para poder acceder de xeito máis rápido aos novos episodios publicados.

Na páxina propia de episodio atopará outro reprodutor HTML5 que lle posibilitará ou ben a escoita por streaming ou ben a \textbf{descarga directa do ficheiro de audio} correspondente. Poderá votar o programa a favor ou en contra (isto repercutirá na popularidade do programa, concepto explicado máis adiante) e deixar un comentario no caso de que esta opción esté habilitada polos administradores do programa.


\subsection{Actividades de produción de contidos}

Son aquelas levadas a cabo polos \textbf{propietarios e administradores} dos programas e emisoras. Un usuario, no seu papel de produtor, poderá engadir ao sistema unha \textbf{nova emisora} cubrindo o formulario habilitado a tal efecto na interface web no que deberá introducir manualmente os datos.

Poderá tamén \textbf{engadir un programa}. Para isto, o usuario só necesitará proporcionarlle ao sistema un enlace ao ficheiro de RSS do podcast que queira engadir e o sistema encargarase de extraer a información tanto dos programas coma dos episodios correspondentes. Este programa e os seus episodios manteranse actualizados xa que o propio backend do sistema comprobará periodicamente se houbo algún cambio no RSS e actualizará a base de datos de xeito acorde.  

Un usuario que posúa un programa ou emisora pode invitar a outro usuario a \textbf{colaborar na xestión} desta ou deste.


\section{Requirimentos funcionais}

Son aqueles que describen as \textbf{accións que o sistema debe efectuar}, isto é: a colección de capacidades e características que o software pon a disposición do usuario. Presentaranse estes en forma de \textbf{historias de usuario} (ver capítulo \ref{chp:metodoloxia} sobre a metodoloxía). 

Se ben é certo que houbo \textbf{encontros con membros de colectivos do terceiro sector} coma a URCM (Unión de radios libres e comunitarias de Madrid) e Cuac FM (A radio comunitaria da Coruña), parte delas están extraídas da \textbf{experiencia propia} coma colaborador nestes medios e coma moderador de comunidades en liña.


\begin{itemize}
	\item Ter un punto de encontro onde ver os programas que están a emitir as emisoras federadas na ReMC (Red de Medios Comunitarios)
	\item Que unha emisora poda acceder aos ficheiros de audio das demais.
	\item Que os \say{autores} dun programa poidan saber en que emisoras está a ser emitido e a que hora.
	\item Que os programas se poidan dar de alta e de baixa.
	\item Poder visualizar información das emisoras e dos programas: Nome, información de emisión, datos de contacto...
	\item Poder subscribirse aos programas.
	\item Poder compartir contido entre usuarios.
	\item Que os ficheiros de audio se asocien de xeito automático a cada emisora.
	\item Que non sexa necesario engadir manualmente os novos ficheiros.
	\item Que os usuarios podan colaborar na xestión das emisoras.
	\item Agrupar os programas por categorías.
\end{itemize}

Eses \say{ficheiros de audio} nos que se pensaba nunha primeira aproximación foron o que finalmente derivou no concepto de \say{episodios de programa} que se explicará máis a fondo no capítulo \ref{chp:disenho} sobre o deseño.

Gran parte destas historias veñen dos usuarios que producen o contido. A medida que avanzaba o deseño e os primeiros pasos da implementación, fóronse perfilando novas historias de usuario pensando máis nos ouvintes:

\begin{itemize}
	\item Que os ouvintes poidan manifestar a súa opinión sobre os programas mediante votos e comentarios.
	\item Que os ouvintes poidan manter unha lista de programas e emisoras favoritas.
	\item Ter ferramentas de procura de contidos.
	\item Recibir recomendacións.  
\end{itemize}  

E tamén outros requirimentos propios da xestión dos contidos.

\begin{itemize}
	\item Crear diferentes roles de administración.
	\item Que o administrador dun programa poida restrinxir a emisión do seu contido.
	\item Que o administrador dun programa poida moderar os comentarios no seu contido.
\end{itemize}  

\section{Requirimentos non funcionais}

Son aqueles que \textbf{especifican as propiedades do sistema} no referente á manexabilidade, seguridade e robustez\cite{softreq}.

\subsection{Autenticación}

O sistema ten que permitir o rexistro de usuarios cun login único e un contrasinal. Dito contrasinal ha de gardarse cifrado na base de datos. A información de sesión da autenticación debe protexerse de manipulacións para evitar impostores.

\subsection{Protección de datos}

Os usuarios deben saber en todo momento que datos persoais deles son recompilados polo sistema e para que fin serán utilizados. Ademáis, débese garantir o seguinte:

\begin{itemize}
	\item \textbf{Dereito ao esquecemento:} Os datos dun usuario han de ser borrados do sistema cando este así o requira.
	\item \textbf{Disponibilidade:} O usuario ha de poder acceder aos seus datos sen posibilidade de ocultación ou demora por parte do sistema. 
\end{itemize}


\subsection{Internacionalización}

O sistema debe ter soporte para a tradución a distintos idiomas. As datas deben gardarse nun formato común e zona horaria UTC(Universal Time Coordinated) para adecuarse na interface ás preferencias do usuario.

\subsection{Rendemento}

A carga de obxectos desde a base de datos debe realizarse de xeito \say{lazy} (preguiceiro) de modo que non se sobrecargue a memoria de forma innecesaria. Isto é especialmente eficiente en comprobacións de existencia de obxectos ou na navegación entre clases relacionadas. 

O tráfico entre o cliente e o servidor debe minimizarse mediante, por exemplo, un sistema de paxinación para aquelas vistas que amosen unha grande cantidade de datos.

\subsection{Concorrencia}

O servidor ten que ser capaz de servir á páxina a un número de usuarios simultáneos aceptable. Isto depende en grande medida do hardware no que o servidor sexa executado. En calquera caso, sendo as emisoras comunitarias colectivos pequenos e baseándonos en datos de visitas á web de Cuac FM, non se espera que o cumprimento deste requisito mereza especial atención.

\subsection{Seguridade}

Os novos datos que se engadan ao sistema han de ser validados para evitar comportamentos maliciosos coma ataques de inxección de código. Cómpre tamén facer validación das URL's para evitar accesos non permitidos. 

\subsection{Adaptabilidade a dispositivos móbiles}

Xa que nos capítulos anteriores mencionamos a importancia dos dispositivos móbiles no crecemento da radio por Internet, a páxina debe ser amigable con eles adaptando a interface cando a pantalla na que se execute sexa pequena.


\subsection{Usabilidade}

O sistema debe poder ser utilizado por usuarios cun perfil non técnico. As ferramentas de entrada e saída de datos han de ser o suficientemente comprensibles e coñecidas para o público xeral.

  \chapter[Deseño]{
  \label{chp:disenho}
  Deseño
}
\minitoc
\newpage

\section{Descrición do funcionamento}

O sistema consiste nunha páxina web con apartados adicados, por un lado, ao que chamamos actividades de consumo, e por outro ás actividades de produción.

\subsection{Actividades de consumo de contidos}

Son aquelas levadas a cabo polos ouvintes das emisoras e os programas presentes no sistema. Un usuario, no seu papel de \say{consumidor}, pode, mediante a interface web, acceder ao catálogo de programas, xa sexa mediante as ferramentas de busca proporcionadas ou mediante as recomendacións que a propia páxina amosa en portada.

Dentro da páxina correspondente a unha emisora, o usuario atopará un reprodutor HTML5 desde o que escoitar a súa emisión en directo mediante streaming. O usuario terá opción de facerse \say{seguidor} da emisora para acceder a ela de xeito máis rápido desde a páxina principal.

Tamén obterá acceso a lista de programas emitidos por dita emisora co seu horario de emisión. Na páxina de cada programa verá a lista completa de episodios dispoñibles e terá a opción de subscribirse para poder acceder de xeito máis rápido aos novos episodios publicados.

Na páxina propia de episodio atopará outro reprodutor HTML que lle posibilitará ou ben a escoita por streaming ou ben a descarga directa do ficheiro de audio correspondente. Poderá votar o programa a favor ou en contra (isto repercutirá na popularidade do programa, concepto explicado máis adiante) e deixar un comentario no caso de que esta opción esté habilitada polos administradores do programa.


\subsection{Actividades de produción de contidos}

Son aquelas levadas a cabo polos donos e administradores dos programas e emisoras. Un usuario, no seu papel de produtor, poderá engadir ao sistema unha nova emisora cubrindo o formulario habilitado a tal efecto na interface web no que deberá introducir manualmente os datos.

Poderá tamén engadir un programa. Para isto, o usuario só necesitará proporcionarlle ao sistema un enlace ao ficheiro de RSS do podcast que queira engadir e o sistema encargarase de extraer a información tanto dos programas coma dos episodios correspondentes. Este programa e os seus episodios manteranse actualizados xa que o propio backend do sistema comprobará periodicamente se houbo algún cambio no RSS e actualizará a base de datos de xeito acorde.  

Un usuario que posúa un programa ou emisora pode invitar a outro usuario a colaborar na xestión desta ou deste.

\section{Arquitectura}

A elección de Django coma framework de desenvolvemento obriga a seguir o patrón Modelo-Vista-Template. Este patrón é unha pequena variación do máis coñecido Modelo-Vista-Controlador, que define 3 capas interconectadas coa fin de separar a lóxica da aplicación(Vista) do almacenamento dos datos(Modelo) e estas dúas, á súa vez, da interface de usuario(Controlador). A vantaxe que ofrecen os \say{templates} de Django é a posibilidade de utilizar os tipos propios do framework para implementar certa lóxica presentacional no renderizado do template.

Neste proxecto, os modelos  definíronse no módulo models.py e as vistas en views.py. Trataranse coma paquetes nos seguintes diagramas. A capa template ven definida no paquete de templates.

\begin{figure}[h]
	\centering
	\includegraphics[scale=0.6,keepaspectratio=true]{./images/project_tree.png}
	\caption{Árbore de módulos do aplicativo web.}
	\label{fig:project_tree}
\end{figure}


\subsection{Capa Modelo}

Á hora de definir esta cuestión, tratáronse de respectar as regras do modelo relacional de bases de datos, porén, fixeronse algunhas excepcións motivadas por esixencia da elección do framework Django:

\begin{itemize}
	\item Claves primarias subrogadas: Django encárgase da identificación das tuplas engadindo ás táboas un campo numérico enteiro de incremento automático. Isto utilizarase en todas as táboas sen excepción.
	
	\item Relación 1-1: Como se ve no diagrama Entidade-Relación da figura \ref{fig:diagrama_er}, existe unha relación 1-1 entre a entidade User e a súa entidade feble UserProfile. O motivo da existencia desta última é que se decidiu utilizar a entidade de usuario nativa de Django, co cal fíxose necesaria unha nova táboa para cubrir os atributos de usuario necesarios especificamente para o proxecto. 
\end{itemize}

No diagrama Entidade-Relación da figura \ref{fig:diagrama_er} pódense ver as táboas empregadas sen incluir aquelas automáticamente xeradas para o correcto funcionamento de Django e Celery a excepción da xa mencionada User. Por motivos de claridade, non se incluíron os atributos agás aqueles adicionais nas táboas correspondentes ás relacións N-N.

A estrutura da BD tradúcese na aplicación ás clases definidas no paquete models.py que se ve na figura \ref{fig:clase_models}. Por claridade, nese diagrama de clases só se incluíron os atributos máis representativos ou explicativos das relacións entre clases.

\begin{figure}[h]
	\centering
	\includegraphics[scale=0.5,keepaspectratio=true]{./images/ER_diagrama.png}
	\caption{Diagrama Entidade-Relación da Base de Datos.}
	\label{fig:diagrama_er}
\end{figure}

\begin{figure}[h]
	\centering
	\includegraphics[scale=0.4,keepaspectratio=true]{./images/class_diagram.png}
	\caption{Diagrama de clases da capa modelo.}
	\label{fig:clase_models}
\end{figure}


\subsubsection{Usuarios}

Os usuarios están definidos por dúas táboas: User e UserProfile que se corresponden coas clases User, do paquete django.contrib.auth.models, e UserProfile, definida polo desenvolvedor e, polo tanto, includida no módulo models.py. Esta segunda considerámola coma feble de User pois a existencia dun perfil está vencellada de xeito ineludible á existencia dun usuario (ver figura \ref{fig:diagrama_er}). Esta entidade auxiliar contén os atributos de usuario: avatar, descrición e localizción. Non se utiliza para máis nada.

A clase User inclúe os atributos necesarios para a autenticación: username, email e password, quedando este último encriptado en base de datos mediante o algoritmo PBKDF2. De entre os campos utilizados para o funcionamento de Django, cómpre destacar is\_staff, que é o que concede acceso do usuario ao panel de administración do aplicativo que provee o propio framework (como se mencionou no apartado \ref{django})

Outros atributos de User reflectidos na figura \ref{fig:clase_models} coma followers ou station\_admins son engadidos automáticamente polo framework ao declarar as relacións coa fin de facilitar a navegación entre clases.


\subsubsection{Programas e episodios}

Entendemos, neste sistema, un programa coma un produto radiofónico do cal se publican entregas periodicamente. Están almacenados na táboa Program da base de datos. Os programas teñen, entre outros atributos, un nome, unha descrición, unha imaxe, un conxunto de categorías (táboa Tag) e un enlace a un ficheiro RSS dado polo usuario.

A subscrición dos usuarios aos programas modelouse coma unha relación N-N. As instancias de Program poden acceder ao conxunto dos seus subscritores mediante o atributo subscribers.

Episodio é como chamamos ás entregas dun programa. Cada un ten, entre outros atributos, un enlace a un ficheiro de audio que será o que se poña a disposición dos usuarios na interface, un resumo, unha imaxe e un conxunto de etiquetas (tags). Un episodio ten que formar parte de 1 e só 1 programa. Se un programa é borrado, os episodios deixan de ter sentido no sistema e son borrados en cascada. Debido ao anterior, considérase a relación destas clases coma unha composición.


\subsubsection{Votos e comentarios}

Nunha primeira aproximación ao deseño da base de datos, entendéronse estas dúas entidades coma simples relacións N-N entre as entidades Episode e User. Porén, decidiuse finalmente que representan conceptos de seu que non perden o sentido semántico fóra da relación e, polo tanto, modeláronse coma entidades (Vote e Comment nos diagramas \ref{fig:diagrama_er} e \ref{fig:clase_models}).

O atributo type de Vote serve para lle dar significado ao voto: Positivo, negativo ou neutro.


\subsubsection{A entidade Station}

Identifícanse coa entidade Station os colectivos aos que se poidan asociar os programas: Emisoras de radio por ondas, emisoras de radio por internet ou canles de podcasting. O atributo mais destacable de Station é o de streaming\_link, que garda o enlace á canle de emisión por internet da emisora que será logo utilizado polo reproductor na web. Este campo pode ser nulo para aqueles colectivos que non dispoñan de emisión en directo.  

A emisión dos programas por parte das emisoras queda modelada coma unha relación N-N entre Program e Station á que chamamos broadcasts. Engadiuse o atributo adicional de schedule\_details, un campo de texto para os detalles de emsión: Horario, periodicidade... A clase correspondente en models.py é Broadcast. Dada a natureza de Station coma aglutinador de programas e dado que non ten sentido a existencia dunha emisión sen emisora, entendeuse que a relación entre Broadcast e Station é de composición. 


\subsubsection{Administración de contido}

Existe unha relación N-N entre os usuarios e as emisoras e unha semellante entre os usuarios e os programas: A de administración. Entendemos esta coma a posibilidade de editar, actualizar e borrar os contidos preexistentes. Isto queda reflectido nas clases ProgramAdmin e StationAdmin. As súas instancias relacionan, respectivamente, aos usuarios cos programas e emisoras que poden administrar e establecen os permisos de administración que posúen, estes últimos, expresados polo atributo type.

Tanto para a administración de emisoras como de Programas, existen dous roles: Propietario e administrador. O propietario ten permisos completos: Edición, actualización, borrado e xestión de administradores. O administrador só ten permisos de edición e actualización.  

Nótese que tal relación de administración non existe no caso dos episodios. Isto débese a que, ao ser engadidos automaticamente segundo a información recibida polo ficheiro RSS do programa (explicado máis en detalle na sección \ref{rss_parser_section}), non son editables. Aquelas opcións que poidan afectar en bloque aos episodios dun programa considéranse xa opcións do programa.

Os superusuarios do sistema, pola súa parte, poden editar e borrar calquera contido mediante ferramentas de Django coma o panel de administración ou o IPython shell.


\subsection{Capa Vista}

A capa vista é na que se atopa a lóxica funcional do aplicativo. Fai uso dos obxectos da capa modelo para, ou ben extraer e compñer os datos que serán enviados ao cliente ou ben recoller os datos enviados desde o cliente para facer as conseguintes modificacións na base de datos. O módulo central desta capa é views.py. Nel, defínense funcións e clases que reciben un obxecto de Django HttpRequest e unha serie de parámetros opcionais e, tras realizar as operacións necesarias, retornan un obxecto HttpResponse.

Un obxecto HttpRequest contén metadatos da petición que se executou desde o cliente: Información da sesión do usuario, tipo de petición (GET, POST...), datos necesarios para os posibles \say{middlewares} e datos que o cliente envía a través dos formularios. 

Os middlewares, en Django, son plugins que alteran globalmente a entrada e saída do procesamento das peticións e as respostas. O framework oferta certa variedade de plugins de serie e a posibilidade de crear middlewares propios. Comentaranse os utilizados no capítulo de Implementación.

Un obxecto HttpResponse contén os datos que han de ser devoltos ao cliente, incluíndo a páxina á que se ha de redirixir ao usuario por causa da petición. Django require gardar os datos nun dicionario clave-valor ao que chama context. Esas claves valerán para acceder aos valores no template á hora de renderizar a resposta.

A decisión de a qué instancia ou función se lle pasa o obxecto HttpRequest tómase en base á url destino enviada polo cliente xunto co propio obxecto. É necesario manter un módulo que relacione as urls coas funcións e clases correspondentes á operación a realizar. Comunmente en Django e tamén neste proxecto, este ficheiro chámase urls.py.

Na figura \ref{fig:vista} amósase un esquema do funcionamento da capa vista. Para unha maior claridade, condensáronse as distintas clases de middleware nun único paso do diagrama. Tamén nesa figura pode verse unha referencia ao obxectos Form de Django que son utilizados neste proxecto. Estes obxectos consisten nunha abstracción a obxecto de Python da información procedente dos formularios despregados no template e dos contedores de dita información (selectores, caixas de texto...). O seu uso é opcional, pero simplifican a definición valores por defecto e a validación dos datos. 

\begin{figure}[h]
	\centering
	\includegraphics[scale=0.6,keepaspectratio=true]{./images/secuencia_vista_v.png}
	\caption{Esquema do funcionamento das vistas.}
	\label{fig:vista}
\end{figure} 


\subsubsection{Engadir contido}
\label{rss_parser_section}

Explicarase en detalle esta vista por ser, a posibilidade de que os usuarios engadan os seus propios programas e os episodios, a parte principal deste proxecto. Un usuario, pode engadir un programa e todos os seus episodios simplemente proporcionándolle ao aplicativo o enlace ao seu ficheiro de RSS.

A función de vista encargada de realizar este traballo recibe o obxecto de request e carga os datos deste nun formulario de Django. Unha vez validado, accede ao ficheiro RSS e extrae os datos do programa e os episodios. Enfrontámonos aquí a unha limitación do proxecto: Non existe un estándar de campos de RSS. Distintos servidores de podcasting poden ter distintos formatos.

Debido a isto, debía deseñarse o sistema de xeito que puidésemos contar con distintos algoritmos de interpretación do RSS e, de cara a unha continuación do desenvolvemento, que engadir novos algoritmos fose sinxelo. De modo que se decidiu aplicar o patrón de deseño \say{estratexia}, como se ve no diagrama \ref{fig:strategy}. A superclase, RSSLinkParser, implementa o método parse\_and\_save, encargado da creación das novas instancias. Esta función utiliza os métodos de lectura da información de programa e episodio, pero deixa a súa implementación ás clases fillas. Actualmente, o proxecto soporta 3 tipos ficheiro RSS dos máis populares entre os usuarios obxectivo.

\begin{figure}[h]
	\centering
	\includegraphics[scale=0.5,keepaspectratio=true]{./images/strategy.png}
	\caption{Patrón Estratexia utilizado para o procesamento de RSS.}
	\label{fig:strategy}
\end{figure}


\subsection{Capa Template}

A capa template define a forma na que os datos obtidos da capa vista serán amosados ao usuario.  [Por Facer]


\section{Actualización dos datos}

[Por Facer]
  \chapter[Implementación]{
  \label{chp:implementacion}
  Implementación
}
\minitoc
\newpage



  \chapter[Probas do sistema]{
  \label{chp:test}
  Probas do sistema
}
\minitoc
\newpage

Neste capítulo explicarase o proceso de probas ao que se someteu o sistema desenvolvido. 

\section{Plan de probas}

Durante as primeiras fases do desenvolvemento utilizouse a técnica \textbf{TDD} (ver capítulo \ref{chp:metodoloxia} sobre metodoloxía) polo que se comezou realizando \textbf{probas de unidade automatizadas} para as clases do modelo de datos e as do módulo rss\_link\_parsers.py.

As \textbf{probas de integración} comezaron á vez que o desenvolvemento das funcións e clases de vista do módulo views.py. Son tamén tests de unidade pero utilizando un simulador de peticións GET e POST.

Finalmente, realizouse un conxunto de \textbf{tests de aceptación} de xeito manual. Utilizáronse para validar o correcto funcionamento da \textbf{interface}.


\section{Probas de unidade}

Son aquelas que verifican o \textbf{correcto funcionamento individual das compoñentes}. Para realizar este tipo de probas, cómpre definir un entorno independente para cada unha de xeito que os resultados das anteriores non inflúan nas posteriores. Os casos de proba están automatizados. Isto fai posible executalos cando se fagan cambios no código co fin de detectar a aparición de efectos non desexados. O código correspondente pode atoparse no ficheiro \textbf{tests.py} incluído no proxecto (ver figura \ref{fig:project_tree})

Utilizouse a biblioteca \textit{django.test} de Django, a cal, mediante o uso do paquete \textit{unittest} estándar de Python, permite a escritura das probas de unidade. Defínense, para isto, conxuntos de casos de proba en forma de clase que ha ser herdeira da \textbf{clase TestCase}, incluída na biblioteca. Confecciónanse, a continuación, as probas en forma de métodos cuxo nome ha levar o prefixo "test".

Para asegurar a independencia entre probas, Django creará unha \textbf{base de datos temporal} que será borrada automaticamente unha vez os tests finalicen, independentemente do seu éxito. Isto implica que cada instancia filla de TestCase teña que popular a base de datos coma paso previo á execución dos tests, podendo isto facerse en cada proba ou declarando un \textbf{método setUp}.


\begin{lstlisting}[language=Python, caption=Probas de unidade dos metodos de Program, label=lst:programtest]
class ProgramModelTests(TestCase): 


	def setUp(self):
	
		rssl = 'http://dummy_link.xml'
		p_image = Image.objects.create(name='p_image1',path='path/to/p_image1')
		
		p1 = Program.objects.create(name='TestProgram1',rss_link=rssl,image=p_image)
		
		u1 = User.objects.create(username='userp1')
		u2 = User.objects.create(username='userp2')
		User.objects.create(username='userp3')
		
		p1.programadmin_set.create(user=u1,type=ADMT_OWNER[0])
		p1.programadmin_set.create(user=u2,type=ADMT_ADMIN[0])
		
		t1 = Tag.objects.create(name='pt1',times_used=1)
		t2 = Tag.objects.create(name='pt2',times_used=2)
		
		p1.tag_set.add(t1)
		p1.tag_set.add(t2)
	
	
	def test_check_user_is_admin(self):
	
		p1 = Program.objects.get(name='TestProgram1')
		u1 = User.objects.get(username='userp1')
		u2 = User.objects.get(username='userp2')
		u3 = User.objects.get(username='userp3')
		
		self.assertTrue(p1.check_user_is_admin(u1,ADMT_OWNER[0]))
		self.assertFalse(p1.check_user_is_admin(u1,ADMT_ADMIN[0]))
		self.assertTrue(p1.check_user_is_admin(u2,ADMT_ADMIN[0]))
		self.assertFalse(p1.check_user_is_admin(u2,ADMT_OWNER[0]))
		self.assertFalse(p1.check_user_is_admin(u3))
	
	
	def test_delete(self):
	
		p1 = Program.objects.get(name='TestProgram1')
		p1_id = p1.id
		img_id = p1.image.id
		
		p1.delete()
		
		p2 = list(Program.objects.filter(pk=p1_id))
		self.assertEqual(p2,[])
		
		img2 = list(Image.objects.filter(pk=img_id))
		self.assertEqual(img2,[])
		
		t1 = Tag.objects.get(name='pt1')
		t2 = Tag.objects.get(name='pt2')
		
		self.assertEqual(t1.times_used,0)
		self.assertEqual(t2.times_used,1)

\end{lstlisting}
 

O código amosado no exemplo \ref{lst:programtest} é o utilizado nas probas de unidade da clase \textbf{Program}. Primeiro, o método \textit{setUp} crea na base de datos temporal o programa \textit{p1}, 3 usuarios dos cales lle asigna 2 coma administradores e máis 2 tags que tamén lle asigna. Nótese que se utiliza un método de creación propio dun obxecto imaxe, iso é posible porque xa ten a súa propia proba de unidade executada anteriormente no mesmo ficheiro. O código xa probado pasa a considerarse seguro e pode ser utilizado nos tests seguintes.

O primeiro test comproba que o método \textit{check\_user\_is\_admin(user,[type])} de Program responda correctamente á pregunta de se o usuario dado é administrador dese programa e cos permisos especificados. O segundo comproba se o método de borrado borra tamén a imaxe asignada e fai decrecer o contador de uso dos tags. Probas semellantes se fixeron para as clases do módulo models.py que posúen métodos propios.


\begin{lstlisting}[language=Python, caption=Probas de unidade dos metodos de ParserIvoox, label=lst:ivooxtest]
class ParserIvooxRSSLPTests(TestCase):


	def initialize_test(self):
	
		RSS_file = TEST_AUX_FILE_PATH + 'hasta-los-kinders_original.xml'
		feed_dict = feedparser.parse(RSS_file)
		u1 = User.objects.create(username='userRSSLP1')
		
		return ParserIvoox(RSS_file,u1),feed_dict
	
	
	def test_get_entry_list(self):
	
		rlp,fd = self.initialize_test()
		el = rlp.get_entry_list(fd)
		
		self.assertIsInstance(el,list)
		self.assertIsInstance(el[0],dict)
	
	
	def test_parse_program(self):
	
		program_web = 'http://www.ivoox.com/podcast-hasta-los-kinders_sq_f14062_1.html'
		rlp,fd = self.initialize_test()
		
		p1 = rlp.parse_program(fd)
		
		self.assertEqual(p1.name,'Hasta Los Kinders')
		self.assertEqual(p1.language,'es-ES')
		self.assertEqual(p1.original_site,program_web)
		self.assertEqual(p1.rss_link_type,IVOOX_TYPE[0])
	
	
	def test_parse_episode(self):
	
		rlp,fd = self.initialize_test()
		
		#Expected info
		exp_titlte = 'CiudadanoKinders: Como hacer un monologo'
		exp_pub_date = datetime.datetime(2010, 9, 17, 20, 45, 24, tzinfo=pytz.utc)
		exp_file = 'http://www.ivoox.com/ciudadanokinders-como-hacer-monologo_mf_368919_feed_1.mp3'
		exp_web = 'http://www.ivoox.com/ciudadanokinders-como-hacer-monologo-audios-mp3_rf_368919_1.html'
		exp_original_id = 'http://www.ivoox.com/368919'
		
		p1 = rlp.parse_program(fd)
		entry_dict = rlp.get_entry_list(fd)[0]
		
		e1 = rlp.parse_episode(entry_dict,p1)
		
		self.assertIsInstance(e1,Episode)
		self.assertEquals(e1.title,exp_titlte)
		self.assertEquals(e1.publication_date,exp_pub_date)
		self.assertEquals(e1.file,exp_file)
		self.assertEquals(e1.original_site,exp_web)
		self.assertEquals(e1.original_id,exp_original_id)
	
	
	# From superclass
	def test_parse_and_save(self):
		
		rlp,_ = self.initialize_test()
		
		p1 = rlp.parse_and_save()
		self.assertIsInstance(p1,Program)
		
		p1_id = p1.id
		p2 = list(Program.objects.filter(pk=p1_id))
		self.assertNotEqual(p2,[])
		
		p2 = p2[0]
		
		self.assertEqual(p2.tag_set.count(),1)
		self.assertEqual(p2.tag_set.all()[0].name,'comedy')
		self.assertIsInstance(p2.image,Image)
		
		
		self.assertEqual(p2.episode_set.count(),20)
		
		e1 = p2.episode_set.filter(title='CiudadanoKinders: Como hacer un monologo')
		self.assertNotEqual(e1,[])
		
		e1 = e1[0]
		self.assertIsInstance(e1,Episode)
		
		self.assertEquals(e1.image,p2.image)
		
		# Clean copied image
		p2.image.delete()
\end{lstlisting}   

No exemplo \ref{lst:ivooxtest} atópase o código que valeu para probar o funcionamento dun dos \textbf{intérpretes de RSS} do módulo rss\_link\_parser.py (ver sección \ref{rss_parser_section}), concretamente o correspondente aos ficheiros co formato de Ivoox.  Tanto con este coma cos outros \say{parsers} probáronse as subrutinas auxiliares e máis os métodos \textit{parse\_program} e \textit{parse\_episode}. No amosado, inclúese tamén o test da función \textit{parse\_and\_save} herdada da superclase.

Para realizar as anteriores probas utilizáronse \textbf{ficheiros RSS reais} descargados a disco local. Dado que o obxectivo último desta clase é a creación de obxectos a partir da información do RSS, as probas enfócanse en comprobar se o gardado en base de datos coincide co esperado.


\begin{lstlisting}[language=Python, caption=Probas de unidade do proceso de actualización de popularidade, label=lst:popdaemon]
class UpdatePopularityDaemonTests(TestCase):


	def setUp(self):
	
		u1 = User.objects.create(username='userUPD1')
		u2 = User.objects.create(username='userUPD2')
		s1 = Station.objects.create(name='RadioTest2')
		
		RSS1 = TEST_AUX_FILE_PATH + 'hasta-los-kinders_original.xml'
		rlp = ParserIvoox(RSS1,u1)
		hlk = rlp.parse_and_save()
		
		hlk_ep1 = hlk.episode_set.all()[0]
		hlk_ep1.vote_set.add(Vote.objects.create(type=LIKE_VOTE[0],user=u1,episode=hlk_ep1))
		
		hlk_ep2 = hlk.episode_set.all()[1]
		hlk_ep2.vote_set.add(Vote.objects.create(type=LIKE_VOTE[0],user=u1,episode=hlk_ep2))
		
		hlk_ep3 = hlk.episode_set.all()[2]
		hlk_ep3.vote_set.add(Vote.objects.create(type=DISLIKE_VOTE[0],user=u1,episode=hlk_ep3))
		hlk_ep3.vote_set.add(Vote.objects.create(type=LIKE_VOTE[0],user=u2,episode=hlk_ep3))
		
		RSS2 = TEST_AUX_FILE_PATH + 'falacalado_podomatic.xml'
		rlp = ParserPodomatic(RSS2,u1)
		fc = rlp.parse_and_save()
		fc.subscribers.add(u1)
		fc.broadcast_set.add(Broadcast.objects.create(station=s1,program=fc,schedule_details='Friday 21:00'))
		
		fc_ep1 = fc.episode_set.all()[0]
		fc_ep1.downloads = 1000
		fc_ep1.save()
	
	
	def test_update_pop_rating_all_programs(self):
		
		hlk = Program.objects.get(name='Hasta Los Kinders')
		fc = Program.objects.get(name="fala calado's podcast")
		
		self.assertEqual(hlk.rating,50)
		self.assertEqual(fc.popularity,0)
		
		#Ignore 365 days limitation
		update_pop_rating_all_programs(days=0)
		
		hlk2 = Program.objects.get(name='Hasta Los Kinders')
		fc2 = Program.objects.get(name="fala calado's podcast")
		
		exp_pop = program_popularity_formula(1,1,0,1000)
		
		self.assertEqual(hlk2.rating,75)
		self.assertEqual(fc2.popularity,exp_pop)
\end{lstlisting}

As funcións que compoñen os procesos executados polo servidor para actualizar os programas (ver sección \ref{daemon_desenho}) tamén foron sometidos a probas. No exemplo \ref{lst:popdaemon} vese o código do test de unidade do proceso que actualiza os campos de \textit{rating} (cualificación) e \textit{popularity} (popularidade) dos programas. Para iso, créase en \textit{setUp} os programas \textit{hlk} e \textit{fc} mediante os parsers de Ivoox e Podomatic (previamente probados) respectivamente. Despois, manipúlanse as características valorables para o rating no primeiro e as valorables para popularity no segundo.

Ao comezar o test, compróbase se os valores de ámbolos dous atributos son os iniciais, logo executase o código de actualización e, finalmente, compróbase se os valores cambiaron da forma esperada.

\section{Probas de integración}

Son aquelas que serven para \textbf{verificar o traballo conxunto de distintas compoñentes}. Estes casos de test, por regra xeral, tratan de probar as conexións entre compoñentes ignorando o funcionamento interno\cite{tests}.

No módulo views.py atópanse as funcións que reciben os datos entrantes e pasan as respostas ao cliente. Para probalas foi necesario o uso da \textbf{clase Client}, dispoñible na biblioteca de Django utilizada para os tests. Esta clase interactúa co sistema a modo de navegador web sinxelo permitindo simular operacións de GET e POST, ver a cadea redirección entre os distintos templates e comprobar que os datos presentes no contexto son os axeitados.

\begin{lstlisting}[language=Python, caption=Fragmento das probas da vista index, label=lst:index]
class IndexViewTests(TestCase):


	@classmethod
	def setUpTestData(cls):
	
		u1 = User.objects.create(username='userIV1')
		
		RSS1 = TEST_AUX_FILE_PATH + 'hasta-los-kinders_original.xml'
		rlp = ParserIvoox(RSS1,u1)
		hlk = rlp.parse_and_save()
		hlk.popularity = 10 # Check if first
		hlk.save()
		
		RSS2 = TEST_AUX_FILE_PATH + 'falacalado_podomatic.xml'
		rlp = ParserPodomatic(RSS2,u1)
		rlp.parse_and_save()
		
		# Total of 10 + 2 programs
		for i in range(1,11):
		
		rssl = 'http://dummy_link_iv_' + str(i) + '.xml'
		pname = 'ProgramIV_' + str(i)
		Program.objects.create(name=pname,rss_link=rssl)
		
		# Create 30 stations
		for i in range(1,31):
		
		sname = 'StationIV_' + str(i)
		Station.objects.create(name=sname)


    def test_view_uses_correct_template(self):

		resp = self.client.get(reverse('rss_feed:index'))
		self.assertEqual(resp.status_code, 200)
		
		self.assertTemplateUsed(resp, 'rss_feed/index.html')


	def test_program_list(self):

		resp = self.client.get(reverse('rss_feed:index'))
		self.assertEqual(resp.status_code, 200)
		
		self.assertEqual( len(resp.context['program_list']),4)
		self.assertTrue(resp.context['program_list'].has_next())
		self.assertFalse(resp.context['program_list'].has_previous())
		
		p1 = resp.context['program_list'][0]
		p2 = Program.objects.get(name='Hasta Los Kinders')
		self.assertEqual(p1,p2)
		
		resp = self.client.get('/rss_feed/?p_page=3')
		self.assertEqual(resp.status_code, 200)
		
		self.assertEqual( len(resp.context['program_list']),4)
		self.assertFalse(resp.context['program_list'].has_next())
		self.assertTrue(resp.context['program_list'].has_previous())
	
\end{lstlisting}


No exemplo \ref{lst:index} vénse algunhas probas realizadas para a vista de \textit{index}, que se corresponde coa páxina principal da aplicación. Créanse primeiro, na función de clase \textit{setUpTestData}, 12 programas e 20 emisoras co obxectivo de comprobar que o contexto está a pasar o número deles correcto, na orde correcta e cos datos axeitados a cerca da paxinación. Isto realízase no segundo test, o do método \textit{test\_program\_list}, só para o caso dos programas (as emisoras tamén se proban, pero incluílo semella redundante). O primeiro test, \textit{test\_view\_uses\_correct\_template}, comproba que se accede á vista index a través do template axeitado.

\begin{lstlisting}[language=Python, caption=Fragmento das probas da vista add\_content, label=lst:add_content]
class AddContentViewTests(TestCase):


	def setUp(self):
	
		u2 = User.objects.create(username='userIV2')
		u2.set_password('12345678A')
		u2.save()
	
	
	def test_login(self):
	
		self.client.login(username='userIV2',password='12345678A')
		resp = self.client.get(reverse('rss_feed:add_content'))
		
		self.assertEqual(resp.status_code, 200)
		self.assertEqual(str(resp.context['user']), 'userIV2')
		
		self.assertTemplateUsed(resp, 'rss_feed/add_content.html')
	
	
	def test_add_station(self):
	
		self.client.login(username='userIV2',password='12345678A')
		
		form_station = {'form_station-name': ['StationIV1'], 'form_station-logo': [''], 'form_station-profile_img': [''], 
		'form_station-broadcasting_method': ['fm'], 'form_station-broadcasting_area': [''], 
		'form_station-broadcasting_frequency': [''], 'form_station-streaming_link': [''], 
		'form_station-description': [''], 'form_station-website': [''], 'form_station-location': ['']}
		
		self.client.post(reverse('rss_feed:add_content'),  form_station)
		
		s1 = Station.objects.get(name='StationIV1')
		
		self.assertIsInstance(s1,Station)
		self.assertEqual(s1.logo,Image.get_default_program_image())
\end{lstlisting}


Aquelas vistas que requiran que o usuario estea identificado tamén poden ser probadas, como se ve no exemplo \ref{lst:add_content}. O primeiro caso proba que o login funciona e que se utiliza o template correcto. O segundo, proba a inserción dunha nova emisora a través do formulario que se enviaría desde a interface web.


\section{Probas de aceptación}

O código puramente de \textbf{frontend} (Javascript, HTML...) foi probado de xeito manual e non automatizado. Tamén se contou cun usuario alleo ao proxecto para realizar un pequeno conxunto de probas.

  \chapter[Conclusións]{
  \label{chp:conclusiones}
  Conclusións
}
\minitoc
\newpage



  
  \appendix
  \chapter[Apéndice]{
  \label{chp:dicionario}
  Dicionario de datos
}
Neste apéndice figura, por orde alfabética, unha descrición dos modelos de datos utilizados. Inclúense so as entidades definidas de forma propia, non aquelas dadas polo framework (por exemplo, User). Do mesmo xeito, non se inclúen os atributos automaticamente definidos pola declaración relacións en Django. 

\textbf{Broadcast:} Clase que representa a relación N-N de emisión entre programas e emisoras.

\begin{longtable}{|p{3cm}|p{3cm}|p{8cm}|}
	\hline
	\rowcolor{gray!50}
	Atributo & Tipo & Descrición\\
	\hline
	@\textbf{id} & Serial & Clave primaria\\
	\hline
	\textbf{program} & ForeignKey & Clave foránea da entidade Program\\
	\hline
	\textbf{station} & ForeignKey & Clave foránea da entidade Station\\	
	\hline
	\textbf{schedule\_details} & Char(100) & Texto do comentario\\
	\hline
\end{longtable}


\textbf{Comment:} Entidade que garda os comentarios que os usuarios deixan nos episodios.

\begin{longtable}{|p{3cm}|p{3cm}|p{8cm}|}
	\hline
	\rowcolor{gray!50}
	Atributo & Tipo & Descrición\\
	\hline
	@\textbf{id} & Serial & Clave primaria\\
	\hline
	\textbf{episode} & ForeignKey & Clave foránea da entidade Episode\\
	\hline
	\textbf{user} & ForeignKey & Clave foránea da entidade User\\	
	\hline
	\textbf{text} & Text & Texto do comentario\\
	\hline
	\textbf{publication\_date} & DateTime & Data de publicación (GMT)\\
	\hline
	\textbf{removed} & Boolean & Marcado coma borrado\\
	\hline
\end{longtable}



\textbf{Episode:} Entidade que garda cada unha das entregas (episodios) dos programas.

\begin{longtable}{|p{3cm}|p{3cm}|p{8cm}|} 
	\hline
	\rowcolor{gray!50}
	Atributo & Tipo & Descrición\\
	\hline
	@\textbf{id} & Serial & Clave primaria\\
	\hline
	\textbf{program} & ForeignKey & Clave foránea da entidade Program\\
	\hline
	\textbf{title} & Char(200) & Título do episodio\\	
	\hline
	\textbf{summary} & Text & Resumo do episodio\\
	\hline
	\textbf{publication\_date} & DateTime & Data de publicación (GMT)\\
	\hline
	\textbf{insertion\_date} & DateTime & Data de inserción na base de datos (GMT)\\
	\hline
	\textbf{file} & URL & Enlace ao ficheiro de audio\\
	\hline
	\textbf{file\_type} & Char(40) & Enlace ao ficheiro de audio\\
	\hline
	\textbf{downloads} & BigInt & Contador de descargas e escoitas do episodio\\
	\hline
	\textbf{original\_id} & Char(200) & Id do episodio no sistema orixinal de almacenamento\\
	\hline
	\textbf{original\_site} & URL & Páxina do episodio no sistema orixinal de almacenamento\\
	\hline
	\textbf{removed} & Boolean & Marcado coma borrado\\
	\hline
	\textbf{image} & ForeignKey & Clave foránea da entidade Image\\
	\hline
	\textbf{votes} & ManyToMany & Referencia ás instancias de Vote relacionadas\\
	\hline
	\textbf{comments} & ManyToMany & Referencia ás instancias de Comment relacionadas\\
	\hline
	
\end{longtable}


\textbf{Program:} Entidade que garda os programas engadidos polos usuarios.

\begin{longtable}{|p{3cm}|p{3cm}|p{8cm}|}
	\hline
	\rowcolor{gray!50}
	Atributo & Tipo & Descrición\\
	\hline
	@\textbf{id} & Serial & Clave primaria\\
	\hline
	\textbf{image} & ForeignKey & Clave foránea da entidade Image\\
	\hline
	\textbf{name} & Char(200) & Nome do programa\\	
	\hline
	\textbf{description} & Text & Texto descritivo sobre o programa\\
	\hline
	\textbf{creation\_date} & DateTime & Data de inserción na base de datos (GMT)\\
	\hline
	\textbf{author\_email} & Email & Correo electrónico do autor do programa\\
	\hline
	\textbf{author} & Char(200) & Autor orixinal extraído do ficheiro RSS\\
	\hline
	\textbf{language} & Char(10) & Código de linguaxe do programa\\
	\hline
	\textbf{rss\_link} & URL & Enlace ao ficheiro RSS do programa\\
	\hline
	\textbf{rss\_link\_type} & Char[ivoox $|$ radioco $|$ podomatic] & Tipo de RSSLinkParser utilizado na súa creación\\
	\hline
	\textbf{rating} & PositiveSmall Integer[0:100] & Cualificación calculada para o programa\\
	\hline
	\textbf{original\_site} & URL & Enlace ao podcast orixinal\\
	\hline
	\textbf{popularity } & Float & Popularidade calculada para o programa\\
	\hline
	\textbf{website} & URL & Páxina web do programa\\
	\hline
	\textbf{sharing\_options} & Char[share\_free $|$ no\_share] & Condicións de compartición\\
	\hline
	\textbf{comment \_options} & Char[enable $|$ disable] & Opción de activar ou desactivar comentarios\\
	\hline
	\textbf{subscribers} &  ManyToMany & Referencia ás instancias de User relacionadas. Representa os subscritores do programa\\
	\hline
	\textbf{admins} &  ManyToMany & Referencia ás instancias de ProgramAdmin relacionadas\\
	\hline
\end{longtable}


\textbf{ProgramAdmin:} Clase que representa a relación N-N de administración entre programas e usuarios.

\begin{longtable}{|p{3cm}|p{3cm}|p{8cm}|}
	\hline
	\rowcolor{gray!50}
	Atributo & Tipo & Descrición\\
	\hline
	@\textbf{id} & Serial & Clave primaria\\
	\hline
	\textbf{program} & ForeignKey & Clave foránea da entidade Program\\
	\hline
	\textbf{user} & ForeignKey & Clave foránea da entidade User\\	
	\hline
	\textbf{type} & Char[owner $|$ admin] & Permisos do usuario sobre o programa\\
	\hline
	\textbf{date} & DateTime & Data na que se concedeu o permiso (GMT)\\
	\hline
\end{longtable}


\textbf{Station:} Entidade que garda os colectivos de emisión (radios por ondas, radios por internet, canles de podcast...)

\begin{longtable}{|p{3cm}|p{3cm}|p{8cm}|}
	\hline
	\rowcolor{gray!50}
	Atributo & Tipo & Descrición\\
	\hline
	@\textbf{id} & Serial & Clave primaria\\
	\hline
	\textbf{logo} & ForeignKey & Clave foránea da entidade Image para o logotipo da emisora\\
	\hline
	\textbf{profile\_img} & ForeignKey & Clave foránea da entidade Image para a imaxe de cabeceira do perfil\\
	\hline
	\textbf{name} & Char(200) & Nome da emisora\\	
	\hline
	\textbf{broadcasting \_method} & Char[RadioFM $|$ RadioAM $|$ RadioDigital $|$ TVChannel $|$ RadioInternet $|$ PodcastingChannel $|$ Others] & Método de emisión dos programas \\
	\hline
	\textbf{broadcasting \_area} & Char(200) & Área de emisión (No caso de RadioFM, RadioAM e RadioDigital)\\
	\hline
	\textbf{broadcasting \_frequency} & Char(50) & Frecuencia de emisión (No caso de RadioFM, RadioAM e RadioDigital)\\
	\hline
	\textbf{streaming\_link} & URL & Enlace á emisión en directo por streaming\\
	\hline
	\textbf{website} & URL & Enlace á páxina web do colectivo\\
	\hline
	\textbf{location} & Char(200) & Localización da emisora\\
	\hline
	\textbf{programs} &  ManyToMany & Referencia ás instancias de Broadcast relacionadas\\
	\hline
	\textbf{admins} &  ManyToMany & Referencia ás instancias de ProgramAdmin relacionadas\\
	\hline
	\textbf{followers} &  ManyToMany & Referencia ás instancias de User relacionadas. Representa os seguidores da emisora.\\
	\hline
\end{longtable}

\textbf{StationAdmin:} Clase que representa a relación N-N de administración entre emisoras e usuarios.

\begin{longtable}{|p{3cm}|p{3cm}|p{8cm}|}
	\hline
	\rowcolor{gray!50}
	Atributo & Tipo & Descrición\\
	\hline
	@\textbf{id} & Serial & Clave primaria\\
	\hline
	\textbf{station} & ForeignKey & Clave foránea da entidade Station\\
	\hline
	\textbf{user} & ForeignKey & Clave foránea da entidade User\\	
	\hline
	\textbf{type} & Char[owner $|$ admin] & Permisos do usuario sobre a emisora\\
	\hline
	\textbf{date} & DateTime & Data na que se concedeu o permiso (GMT)\\
	\hline
\end{longtable}	


\textbf{Tag:}  Entidade que garda os etiquetas de categoría dadas polos ficheiros RSS.

\begin{longtable}{|p{3cm}|p{3cm}|p{8cm}|}
	\hline
	\rowcolor{gray!50}
	Atributo & Tipo & Descrición\\
	\hline
	@\textbf{id} & Serial & Clave primaria\\
	\hline
	\textbf{name} & Char(50) & Nome do tag en minúsculas. Ten que ser único\\
	\hline
	\textbf{times\_used} & Positive IntegerField & Cantidade de veces presente en programas e episodios\\	
	\hline
	\textbf{programs} & ManyToMany & Referencia ás instancias de Program relacionadas\\
	\hline
	\textbf{episodes} & ManyToMany & Referencia ás instancias de Episode relacionadas\\
	\hline
\end{longtable}

\pagebreak
\textbf{UserProfile:}  Entidade que extende a clase User para asignarlle novos atributos.

\begin{longtable}{|p{3cm}|p{3cm}|p{8cm}|}
	\hline
	\rowcolor{gray!50}
	Atributo & Tipo & Descrición\\
	\hline
	@\textbf{id} & Serial & Clave primaria\\
	\hline
	\textbf{user} & OneToOne & Referencia á instancia de User que extende\\
	\hline
	\textbf{description} & Text & Texto de presentación do usuario\\	
	\hline
	\textbf{avatar} & ForeignKey & Clave foránea da entidade Image\\
	\hline
	\textbf{location} & Char(100) & Localización do usuario\\
	\hline
\end{longtable}


\textbf{Vote:}  Entidade que representa os votos cos que os usuarios cualifican os episodios.

\begin{longtable}{|p{3cm}|p{3cm}|p{8cm}|}
	\hline
	\rowcolor{gray!50}
	Atributo & Tipo & Descrición\\
	\hline
	@\textbf{id} & Serial & Clave primaria\\
	\hline
	\textbf{user} & ForeignKey & Clave Foránea da entidade User\\
	\hline
	\textbf{episode} & ForeignKey & Clave foránea da entidade Episode\\
	\hline
	\textbf{location} & Char(100) & Localización do usuario\\
	\hline
	\textbf{date} & DateTime & Data na que se concedeu o permiso (GMT)\\
	\hline
\end{longtable}


  \chapter[Apéndice]{
  \label{chp:dicionario}
  Licenza
}

Este apéndice ten o propósito de determinar a licenza baixo a que se distribúen este proxecto e maila súa documentación. Un dos obxectivos a priori era que o software resultante fose compatible coa definición de software libre que da a Free Software Foundation, segundo a cal, un programa ten que cumprir o que comunmente se chama \say{As catro liberdades esenciais}\cite{licenza}:

\begin{itemize}
	\item \textbf{Liberdade 0:} Liberdade de executar o programa con calquera propósito desexable.
	\item \textbf{Liberdade 1:} Liberdade para estudar o funcionamento do programa e de modificalo sen limitacións. Para garantir esta liberdade, o código fonte debe estar dispoñible ao acceso público.
	\item \textbf{Liberdade 2:} Liberdade de redistribución de copias do programa orixinal. 
	\item \textbf{Liberdade 3:} Liberdade de distribución de copias de versións modificadas do software sempre e cando ese código modificado estea á súa vez dispoñible ao público.
\end{itemize} 

\section{Licenzas das dependencias do proxecto}

A continuación móstrase unha táboa das dependencias do software deste proxecto. Centrarémonos en dúas características:

\begin{itemize}
	\item Liberdade: Se a licenza é compatible coa definición de software libre vista con anterioridade.
	\item Copyleft: Dise daquela licencia que esixa preservar as súas liberdades na distribución do produto ou derivados. Un software con licenza libre non copyleft podería ser utilizado noutro software de natureza privativa.
\end{itemize} 
\break

\begin{longtable}{|p{3cm}|p{1.5cm}|p{5.5cm}|p{1cm}|p{1.5cm}|}
	\hline
	\rowcolor{gray!50}
	Dependencia & Versión & Licenza & Libre & Copyleft \\
	\hline
	\textbf{Anaconda} & 4.4.0 (64bits) & BSD (Berkeley Software Distribution) & Si & Non \\
	\hline
	\textbf{Django} & 1.11 & BSD & Si & Non \\
	\hline
	\textbf{psycopg2} & 2.7.1 & LGPL (GNU Lesser General Public License) & Si & Si \\	
	\hline
	\textbf{django-bootstrap4} & 0.0.6 & Apache Software License 2.0 & Si & Si\\
	\hline
	\textbf{celery} & 4.1.0 & BSD & Si & Non \\
	\hline
	\textbf{django-celery-results} & 1.0.1 & BSD & Si & Non \\
	\hline
	\textbf{django-celery-beat} & 1.1.1 & BSD & Si & Non \\
	\hline
	\textbf{django-static-jquery} & 2.1.4 & MIT (Massachusetts Institute of Technology) & Si & Non \\
	\hline
	\textbf{gettext} & 0.19.8.1-3 & GPLv3 (GNU General Public License) & Si & Si \\
	\hline
\end{longtable}


\section{Conclusión}

O software creado neste proxecto publícase baixo a licenza \textbf{GPLv3}. Esta foi orixinalmente creada para o proxecto GNU pola propia Free Software Foundation, sendo a versión 3, publicada en 2007, a máis recente. Trátase dunha licencia copyleft, o cal significa que a distribución das copias deste software e calquera traballo derivado han de ser tamén un proxecto de software libre. Esta licenza permite non so que, no futuro, outros desenvolvedores podan colaborar na mellora e mantemento deste software, senón tamén que outros proxectos o reutilicen.

A documentación, é dicir, a presente memoria, queda publicada baixo licenza \textbf{GFDL} (GNU Free Documentation License). É unha licenza libre creada tamén pola FSF para o proxecto GNU. Este feito implica que o presente documento pode ser copiado e modificado por quen o desexe. Ao ser GFDL unha licenza copyleft, a redistribución de copias e obras derivadas desta documentación está suxeita ao mantemento dos termos da licencia por parte das mesmas.


  % Glossary
  %\printglossary[title=Glosario,toctitle=Glosario]
  %\printglossary[type=\acronymtype,title=Acrónimos,toctitle=Acrónimos]
  % Bibliography
  %\nocite{*}    % incluir referencias no citadas.
  %\bibliographystyle{plainnat}
  \bibliographystyle{unsrtnat}
  %\bibliographystyle{ksfh_nat}
  %\bibliographystyle{dcu}  % Books and web
  \bibliography{pfc}

  
\end{document}

