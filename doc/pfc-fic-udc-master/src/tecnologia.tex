\chapter[Tecnoloxías Utilizadas]{
  \label{chp:tecnologia}
  Tecnoloxía
}
\minitoc
\newpage

Este capítulo contempla a base tecnolóxica utilizada na última versión do proxecto até a data
de entrega desta memoria. Os conceptos listados nesta sección están suxeitos a posibles cambios
por mor dos futuros traballos de actualización e despregue nun entorno de produción que se pretenden
efectuar. 

Á hora de elixir o uso destas ferramentas valorouse a súa natureza open source xa que se pretende que o
resultado dispoña dunha licencia de software libre compatible coa definición da Free Software
Foundation. 

Tamén se valoraron, dado que nos capítulos anteriores insistiuse na importancia dos dispositivos móviles
no consumo da radio a través de Internet, as posibilidades de ditas ferramentas de ofrecer un bo resultado
en distintos dispositivos e distintas plataformas software. 


\section{Linguaxes}

\subsection{Python}

Python é unha linguaxe de propósito xeral de alto nivel. Trátase dunha linguaxe interpretada polo que é necesario 
ter un intérprete para executar o código. Ao ser o intérprete unha capa intermedia de software entre o programa
e o sistema, Python é unha linguaxe fácilmente portable entre dispositivos de distinta natureza.

A súa sintaxe baseada na indentación está pensada para favorecer a comprensión entre distintos desenvolvedores e 
facilitar o mantemento do código. 

O seu sistema de tipado implícito e a súa riqueza de bibliotecas para executar comandos de consola de xeito programático
fan que sexa moitas veces descrito coma \say{unha linguaxe de scripting orientada a obxectos}\cite{python1}, o cal serve coma 
mostra da súa flexibilidade, un dos motivos polo que foi elixida para este proxecto.

\subsubsection{Anaconda}

Anaconda é unha distribución de Python normalmente utilizada para fins estatísticos e de análise de datos. Inclúe tamé  




\section{Django Framework}

O proxecto Django nace no ano 2005 coma un conxunto de ferramentas Python para o desenvolmento web publicadas
baixo licencia BSD. A motivación dos seus creadores, Simon Wilson e Adrian Holovaty, naquel tempo programadores
nun medio periodístico, era a homoxeneización e a reutilización de código. Isto daba resposta á necesidade de reducir
o alto custo que supón manter código ad-hoc para cada novo artigo ou función, especialmente nunha páxina que
requira unha constante actualización de contidos coma a dun diario online. É por iso que se adoita utilizar o 
termo \say{newsroom schedule} cando se fala da rapidez de desenvolvemento que permite o framework\cite{django1}.  

Django ofrece un conxunto de bibliotecas que dan soporte ás mecánicas máis comúns do desenvolvemento
web:

\begin{itemize}
	
	\item Persistencia: Django permite crear a estrutura da base de datos sen necesidade de escribir SQL. Isto 
	conséguese mediante a superclase Model da librería \say{db} de Django. Ao declarar unha clase de Python coma
	extensión de \say{Model}, a biblioteca ocúpase automáticamente de manter a coherencia entre as súas instancias en 
	memoria e as táboas da base de datos. 
	
	\item Peticións web: A lectura e a interpretación destas peticións son levadas a cabo polo conxunto de bibliotecas
	de HTTP do framework, encapsulando as peticións e as respostas en obxectos para formar un estándar sinxelo e garantir
	a correcta formación das mesmas.
	
	\item Enrutamento e Validación: O framework ofrece un estándar para asignar as distintas vistas definidas no código
	ás URL's desexadas. Os posibles formularios presentes nesas vistas poden ser abstraídos en obxectos de Python 
	simplificando a validación dos datos introducidos polos usuarios. A utilización destes sistemas é non obstante, 
	opcional, puidendo o programador optar por utilizar sinxelos formularios HTML no caso de que isto axilizase 
	o desenvolvemento.
	
	\item Sistema para embeber datos dinámicos no HTML: O chamado \say{template system} de Django marca uns procedementos
	sinxelos para acceder aos datos xenerados no código. Tamén ofrece ferramentas para implementar certa lóxica
	presentacional.
	
	\item Interface de administración nativa: Por defecto, este framework dispón dun panel de administración de acceso
	reservado aos superusuarios que evita os accesos directos á base de datos no caso de que un administrador necesite
	facer cambios manuais nos datos.
	
	\item Soporte por defecto para a autenticación e autorización: Django permite ao desenvolvedor invocar a clase 
	nativa Usuario e efectuar con ela accións de rexistro, autenticación e concesión/revocación de permisos sen máis
	programación. Se ben, como se explica máis adiante, neste proxecto extendeuse esa clase Usuario e creouse unha
	xerarquía de permisos propia.

\end{itemize}   


Para este proxecto utilizouse concretamente Django 1.11, a versión estable máis avanzada no tempo en que se comezou o 
traballo. Django sufriu unha actualización importante desde entón que rematou coa publicación de Django 2.0 en Decembro do pasado ano 2017. Estas dúas versións non son totalmente compatibles de modo que un upgrade implicaría cambios no 
código. Porén, a Django Software Foundation, responsable do mantemento do framework, non prevee retirar o soporte á
versión utilizada nun futuro próximo e a versión de Python utilizada sí é soportada por Django 2\cite{django2}. De modo que 
se considerou suficiente a actualidade desta versión.    

Ademáis do mencionado anteriormente, confiouse en Django para este proxecto coma un framework veterán utilizado en sitios
populares coma Instagram, Disqus, Pinterest ou Open Knowledge Foundation.

