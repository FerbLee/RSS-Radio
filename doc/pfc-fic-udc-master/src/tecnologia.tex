\chapter[Tecnoloxías Utilizadas]{
  \label{chp:tecnologia}
  Tecnoloxía
}
\minitoc
\newpage

Este capítulo contempla a base tecnolóxica utilizada na última versión do proxecto até a data
de entrega desta memoria. Os conceptos listados nesta sección están suxeitos a posibles cambios
por mor dos futuros traballos de actualización e despregue nun entorno de produción que se pretenden
efectuar. 

Á hora de elixir o uso destas ferramentas valorouse a súa natureza open source xa que se pretende que o
resultado dispoña dunha licencia de software libre compatible coa definición da Free Software
Foundation. 

Tamén se valoraron, dado que nos capítulos anteriores insistiuse na importancia dos dispositivos móviles
no consumo da radio a través de Internet, as posibilidades de ditas ferramentas de ofrecer un bo resultado
en distintos dispositivos e distintas plataformas software. 


\section{Linguaxes}

\subsection{Python}
\label{python}
Python é unha linguaxe de propósito xeral de alto nivel. Trátase dunha linguaxe interpretada polo que é necesario 
ter un intérprete para executar o código. Ao ser o intérprete unha capa intermedia de software entre o programa
e o sistema, Python é unha linguaxe fácilmente portable entre dispositivos de distinta natureza.

A súa sintaxe baseada na indentación está pensada para favorecer a comprensión entre distintos desenvolvedores e 
facilitar o mantemento do código. 

O seu sistema de tipado implícito e a súa riqueza de bibliotecas para executar comandos de consola de xeito programático
fan que sexa moitas veces descrito coma \say{unha linguaxe de scripting orientada a obxectos}\cite{python1}, o cal serve coma 
mostra da súa flexibilidade, un dos motivos polo que foi elixida para este proxecto.

\subsubsection{Anaconda}

Anaconda é unha distribución de Python, libre e de código aberto (licenza BSD), utilizada normalmente para análise estatística e machine learning. Inclúe, ademáis dunha completa instalación do intérprete de Python, un rico conxunto de librarías de uso común e o xestor de paquetes conda, sendo esas dúas últimas melloras o motivo primordial polo que foi a elección para este proxecto. Conda utiliza os paquetes da comunidade de Conda Forge \cite{anaconda1}

Anaconda é responsabilidade de Anaconda Incorporated, antigo Continuum Analytics.

A versión utilizada foi a Anaconda 4.4.0 de 64 bits para Python 3, a máis nova no momento de comezar o traballo. Inclúe o intérprete para Python versión 3.6.1.


\subsection{HTML}

HTML (acrónimo de HyperText Marckup Language) é a linguaxe estándar para a creación da estrutura das páxinas web. Os 
elementos presentes na web declaranse mediante bloques representados por etiquetas(tags), de ahí que reciba o nome de
\say{marckup language} (linguaxe de marcado). Estas etiquetas son interpretadas polos navegadores para renderizar o 
contido da páxina\cite{html1}. Ao ser un estándar tan lonxevo, consolidado e popular, non se consideraron alternativas 
para este proxecto.

Dado á natureza multimedia da web realizada, utilizouse HTML5. Esta quinta revisión do estándar inclúe novas etiquetas 
para o tratamento das imaxes e dos contidos de audio e vídeo sen necesidade de utilizar outras tecnoloxías complementarias, 
simplificando así o desenvolvemento e o mantemento posterior do portal. Desde decembro do pasado ano 2017, a versión 5.2 é a última estable recomendada polo World Wide Web Consortium(W3C)\cite{html_w3c}.


\subsection{CSS}

CSS (Acrónimo de Cascading Style Sheets) ou \say{Follas de estilo} é unha linguaxe empregada para definir a estética dos elementos definidos anteriormente no código HTML: A súa cor, fonte, posición... Ao separar o estilo da estrutura, favorécese a reutilización de código xa que un mesmo ficheiro de CSS pode ser utilizado en diversas páxinas ao mesmo tempo. CSS permite tamén adaptar o contido das páxinas a dispositivos de distinto tamaño\cite{css_w3c}.

\subsubsection{CSS Grid}

O módulo de \say{Grid} úsase para definir un deseño de interface gráfica consistente nunha cuadrícula de dúas dimensións en CSS. Nun modelo deste tipo, declárase un conxunto de elementos no HTML coma pertencentes a un \say{grid container}(contedor da grella ou da cuadrícula) e, á súa vez, zonas fillas que se posicionarán dentro dese contedor dependendo das características coas que este último fose definido na folla de estilos.

Unha cela nun contedor pode, á súa vez, consistir noutro contedor, permitindo así deseños asimétricos. O tamaño das celas pode ser fixo, relativo á páxina ou automático dependendo do contido da cela. Por todo o dito, este módulo proporciona un nivel de flexibilidade superior ao que poderíamos obter utilizando outras alternativas populares coma CSS Flexbox. 

O nivel primeiro de CSS Grid non acadou aínda, na data de entrega desta memoria, o status de \say{recomendación} da  World Wide Web Consortium(W3C)\cite{css_grid_w3c}, porén, xa é soportado polas últimas versións dos navegadores máis populares\cite{css_grid_w3c2} polo que non se considerou un risco utilizalo neste proxecto.

\subsection{JavaScript}

JavaScript é unha linguaxe de scripting interpretada e de alto nivel. Utilízase principalmente (e tamén neste proxecto) coma unha ferramenta para mellorar a interactividade da interface web ao permitir executar código orientado a eventos no lado do cliente, aforrando recargas da páxina e incluso accesos innecesarios á base de datos xa que permite o uso do disco local mediante cookies. 

A pesares do seu nome, non garda relación algunha coa linguxe Java e serven a propósitos claramente distintos. O núcleo desta linguaxe está regulado polo estándar ECMAScript\textregistered, na súa 8\textsuperscript{a} versión\cite{ecma} no momento de entregar esta memoria. 

Como se comentou no apartado \ref{python}, que sexa interpretada implica a necesidade dun intérprete para executar o código. Ese intérprete, comunmente chamado JavaScript Engine preséntase embebido nos navegadores web. Non obstante, non éxiste unha única implementación senón que distintos navegadores presentan distintas versións. Para o desenvolvemento deste proxecto utilizáronse os navegadores Mozilla Firefox 57.0.1 e Google Chrome 66.0, cuxos motores de JavaScript son SpiderMonkey e V8 respectivamente\cite{javascript1}, ambos os dous libres e de código aberto.

Pese a que o seu uso nos navegadores continúa a ser a principal razón de ser desta linguaxe, tamén se utiliza a día de hoxe noutro tipo de produtos coma Node.js(para correr JavaScript no lado do servidor) ou Apache Couch DB (Para o manexo de bases de datos na nube)\cite{javascript2} 


\subsection{SQL}


\subsection{XML}

\section{Django Framework}

O proxecto Django nace no ano 2005 coma un conxunto de ferramentas Python para o desenvolmento web publicadas
baixo licencia BSD. A motivación dos seus creadores, Simon Wilson e Adrian Holovaty, naquel tempo programadores
nun medio periodístico, era a homoxeneización e a reutilización de código. Isto daba resposta á necesidade de reducir
o alto custo que supón manter código ad-hoc para cada novo artigo ou función, especialmente nunha páxina que
requira unha constante actualización de contidos coma a dun diario online. É por iso que se adoita utilizar o 
termo \say{newsroom schedule} cando se fala da rapidez de desenvolvemento que permite o framework\cite{django1}.  

Django ofrece un conxunto de bibliotecas que dan soporte ás mecánicas máis comúns do desenvolvemento
web:

\begin{itemize}
	
	\item Persistencia: Django permite crear a estrutura da base de datos sen necesidade de escribir SQL. Isto 
	conséguese mediante a superclase Model da librería \say{db} de Django. Ao declarar unha clase de Python coma
	extensión de \say{Model}, a biblioteca ocúpase automáticamente de manter a coherencia entre as súas instancias en 
	memoria e as táboas da base de datos. 
	
	\item Peticións web: A lectura e a interpretación destas peticións son levadas a cabo polo conxunto de bibliotecas
	de HTTP do framework, encapsulando as peticións e as respostas en obxectos para formar un estándar sinxelo e garantir
	a correcta formación das mesmas.
	
	\item Enrutamento e Validación: O framework ofrece un estándar para asignar as distintas vistas definidas no código
	ás URL's desexadas. Os posibles formularios presentes nesas vistas poden ser abstraídos en obxectos de Python 
	simplificando a validación dos datos introducidos polos usuarios. A utilización destes sistemas é non obstante, 
	opcional, puidendo o programador optar por utilizar sinxelos formularios HTML no caso de que isto axilizase 
	o desenvolvemento.
	
	\item Sistema para embeber datos dinámicos no HTML: O chamado \say{template system} de Django marca uns procedementos
	sinxelos para acceder aos datos xenerados no código. Tamén ofrece ferramentas para implementar certa lóxica
	presentacional.
	
	\item Interface de administración nativa: Por defecto, este framework dispón dun panel de administración de acceso
	reservado aos superusuarios que evita os accesos directos á base de datos no caso de que un administrador necesite
	facer cambios manuais nos datos.
	
	\item Soporte por defecto para a autenticación e autorización: Django permite ao desenvolvedor invocar a clase 
	nativa Usuario e efectuar con ela accións de rexistro, autenticación e concesión/revocación de permisos sen máis
	programación. Se ben, como se explica máis adiante, neste proxecto extendeuse esa clase Usuario e creouse unha
	xerarquía de permisos propia.

\end{itemize}   


Para este proxecto utilizouse concretamente Django 1.11, a versión estable máis avanzada no tempo en que se comezou o 
traballo. Django sufriu unha actualización importante desde entón que rematou coa publicación de Django 2.0 en Decembro do pasado ano 2017. Estas dúas versións non son totalmente compatibles de modo que un upgrade implicaría cambios no 
código. Porén, a Django Software Foundation, responsable do mantemento do framework, non prevee retirar o soporte á
versión utilizada nun futuro próximo e a versión de Python utilizada sí é soportada por Django 2\cite{django2}. De modo que 
se considerou suficiente a actualidade desta versión.    

Ademáis do mencionado anteriormente, confiouse en Django para este proxecto coma un framework veterán utilizado en sitios
populares coma Instagram, Disqus, Pinterest ou Open Knowledge Foundation.

