\chapter[Introducción]{
  \label{chp:introduccion}
  Introdución
}
\minitoc
\newpage


\section{Marco do proxecto}

A radiodifusión tradicional, entendendo esta coma a retransmisión de contidos de audio a través de ondas analóxicas, presenta a día de hoxe unha serie de limitacións. A máis importante, se cadra, é o feito de que a demanda de frecuencias é superior ao que o espectro radioeléctrico pode ofrecer. Cómpre, por iso, a existencia de unha autoridade que outorgue licenzas de emisión sendo ditas institucións, na nosa sociedade, a Secretaría de Estado de Telecomunicaciones y para la Sociedad de la Información e máis a Secretaría Xeral de Medios \cite{ley_audiovisual}. Isto conleva a imposibilidade de emitir contidos por parte de aqueles que ou ben non poidan facer fronte á inversión que unha licenza supón ou ben non lles fose outorgada.

A medida que o acceso a Internet se fai máis cotián, a emisión por streaming eríxese coma solución aos problemas da radio en FM. A través deste medio, unha emisora pode emitir contidos sen necesidade de licenzas, acadando, ademáis, unha cobertura global de xeito centralizado, sen necesidade de emitir mediante cadeas de radio enlace, como é costume nas grandes emisoras en FM do Estado Español, coa inversión en infrastruturas que isto conleva.

Internet permite non só a emisión en directo mediante streaming, senón tamén o acceso a contidos baixo demanda. Isto propiciou a aparición do podcasting a mediados da década dos 2000. A aparición de ferramentas de uso diario que permiten un accesso fácil a estes contidos está a amplificar o impacto da radio por Internet e o podcasting e a afectar ao comportamento dos usuarios: Actualmente, un 7.5\% dos ouvintes escoitan a radio por Internet (6.1\% por streaming, 1.6\% en diferido ou podcast), aínda lonxe dos ouvintes de FM, porén máis do dobre dos de Onda Media.





\section{Marco do proxecto}

\section{Motivación}

\section{Obxectivos}